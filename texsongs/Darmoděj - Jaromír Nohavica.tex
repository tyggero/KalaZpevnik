\beginsong{Darmoděj}[by={Jaromír Nohavica}]
\num
\[Am]{Šel včera} městem \[Em]{muž a} šel po hlavní \[Am]třídě \[Em]    \[Am]
Šel včera městem \[Em]{muž a} já ho z okna \[Am]viděl \[Em]
\[C]{Na flétnu} chorál \[G]hrál, znělo to jako \[Am]zvon
A byl v tom všechen \[Em]{žal, ten} krásný dlouhý \[F]tón
A já jsem náhle \[F\shrp{}\didx{dim}]{věděl: Ano,} to je \[E\hidx{7}]{on, to} je \[Am]{on}
\fin
\chordsoff
\num
Vyběh jsem do ulic jen v noční košili
V odpadcích z popelnic krysy se honily
A v teplých postelích lásky i nelásky
Tiše se vrtěly rodinné obrázky
A já chtěl odpověď na svoje otázky, otázky
\fin
\chor
\chordson
\[Am]{Na na} na \[Em]na\ldots{} \[C]  \[G]  \[Am]    \[F]  \[F\shrp{}\didx{dim}]      \[E\hidx{7}]
\cl
\num
Dohnal jsem toho muže a chytl za kabát
Měl kabát z hadí kůže, šel z něho divný chlad
A on se otočil a oči plné vran
A jizvy u očí, celý byl pobodán
A já jsem náhle věděl kdo je onen pán~-- onen pán
\fin
\num
Celý se strachem chvěl, když jsem tak k němu došel
A v ústech flétnu měl od Hieronyma Bosche
Stál měsíc nad domy jak čírka ve vodě
Jak moje svědomí, když zvrací v záchodě
A já jsem náhle věděl: to je Darmoděj, můj Darmoděj
\fin
\chor
\chordson
\[Am]{Můj Darmo}\[Em]{děj, vaga}\[C]bund osu\[G]{dů a} lásek
\[Am]{Jenž prochá}\[F]{zí všemi} \[F\shrp{}\didx{dim}]{sny,  ale} dnům \[E\hidx{7}]vyhýbá se
\cl
\freev
Můj Darmoděj, krásné zlo, jed má pod jazykem
Když prodává po domech jehly se slovníkem
\cl
\num
Šel včera městem muž, podomní obchodník
Šel, ale nejde už, krev skápla na chodník
Já jeho flétnu vzal a zněla jako zvon
A byl v tom všechen žal, ten krásný dlouhý tón
A já jsem náhle věděl: ano, já jsem on, já jsem-
\fin
\chor
Váš Darmoděj, vagabund osudů a lásek
Jenž prochází všemi sny, ale dnům vyhýbá se
Váš Darmoděj, krásné zlo, jed mám pod jazykem
Když prodávám po domech jehly se slovníkem
\cl
\endsong




