\documentclass[a5paper,10pt]{book}
\usepackage[chorded]{songs}
\usepackage[czech]{babel}
\usepackage[utf8]{inputenc}
\usepackage{fullpage}
\usepackage{ifthen}
\usepackage{textpos}
\usepackage[T1]{fontenc}
\usepackage{lmodern}
\usepackage[unicode=true,hidelinks]{hyperref}

\def \nempty {999}
\def \nchorus {1000}
\def \nchorusi {1001}
\def \nchorusii {1002}
\def \naverse {1101}
\def \nbverse {1102}
\def \ncverse {1103}
\def \nintro {1201}
\def \nsolo {1202}
\def \nbridge {1203}
\def \nrecite {1204}
\newcounter{oldversenum}
\newindex{titleidx}{titleidx}
\newauthorindex{authidx}{authidx}
\songcolumns{1}
\songpos{3}
\setlength{\versenumwidth}{0.75cm}
\addtolength{\hoffset}{-15pt}
\addtolength{\voffset}{-35pt}
\addtolength{\textheight}{100pt}
\addtolength{\textwidth}{35pt}
\setlength{\sbarheight}{0pt}
\setlength{\versesep}{10pt plus 8pt minus 2pt}
% \pagestyle{empty}
\renewcommand{\printversenum}[1]{\ifthenelse{\equal{\theversenum}{\nempty}}{}{\small\sffamily\ifthenelse{\equal{\theversenum}{\nchorus}}{R.}{\ifthenelse{\equal{\theversenum}{\nchorusi}}{R1.}{\ifthenelse{\equal{\theversenum}{\nchorusii}}{R2.}{\ifthenelse{\equal{\theversenum}{\naverse}}{A.}{\ifthenelse{\equal{\theversenum}{\nbverse}}{B.}{\ifthenelse{\equal{\theversenum}{\ncverse}}{C.}{\ifthenelse{\equal{\theversenum}{\nintro}}{Intro }{\ifthenelse{\equal{\theversenum}{\nsolo}}{Solo }{\ifthenelse{\equal{\theversenum}{\nbridge}}{M.}{\ifthenelse{\equal{\theversenum}{\nrecite}}{Rec.}{\theversenum.}}}}}}}}}}}\ }
\renewcommand{\printchord}[1]{\bf\sffamily#1}
\renewcommand\musicnote[1]{\ifchorded\vspace{-5pt}\textnote{#1}\vspace{-5pt}\fi}

\afterpreludeskip=0pt plus 5pt
\beforepostludeskip=0pt plus 5pt
\baselineadj=2pt plus 1pt minus 2pt
\renewcommand{\clineparams}{
  \baselineskip=10pt
  \lineskiplimit=1pt
  \lineskip=1pt
}


\renewcommand{\capo}[1]{\iftranscapos\transpose{#1}\else\musicnote{\sffamily{}Capo #1\normalfont}\fi}

\newcommand{\musicblock}[1]{\ifchorded #1 \fi}
% \newcommand{\ctabblock}[1]{\begin{textblock}{10}(5,-0.5)#1\end{textblock}}

\renewcommand{\idxtitlefont}{\small\normalfont}
\renewcommand{\idxlyricfont}{\small\normalfont}
\renewcommand{\idxauthfont}{\small\normalfont}
\newcommand{\authfont}{\itshape\sffamily}
\renewcommand{\stitlefont}{\bf\Large\sffamily}
%\renewcommand{\printsongnum}[1]{\bf\LARGE\sffamily#1}
\renewcommand{\printsongnum}[1]{}
\renewcommand{\snumbgcolor}{white}
\renewcommand{\extendprelude}{\vspace{2pt}\footnotesize\showauthors\showrefs}
\newcommand{\fadeout}{\footnotesize\sffamily to fade out \normalfont\normalsize}

\makeatletter

\renewcommand\showauthors{%
  \setbox\SB@box\hbox{\authfont\sfcode`.\@m\songauthors\normalfont}%
  \ifdim\wd\SB@box>\z@\unhbox\SB@box\par\fi%
}

% \newcommand\textsubscript[1]{\@textsubscript{\selectfont#1}}
% \def\@textsubscript#1{{\m@th\ensuremath{_{\mbox{\fontsize\sf@size\z@#1}}}}}
% \newcommand\textbothscript[2]{%
  % \@textbothscript{\selectfont#1}{\selectfont#2}}
% \def\@textbothscript#1#2{%
  % {\m@th\ensuremath{%
    % ^{\mbox{\fontsize\sf@size\z@#1}}%
    % _{\mbox{\fontsize\sf@size\z@#2}}}}}
% \def\@super{^}\def\@sub{_}

% \catcode`^\active\catcode`_\active
% \def\@super@sub#1_#2{\textbothscript{#1}{#2}}
% \def\@sub@super#1^#2{\textbothscript{#2}{#1}}
% \def\@@super#1{\@ifnextchar_{\@super@sub{#1}}{\textsuperscript{#1}}}
% \def\@@sub#1{\@ifnextchar^{\@sub@super{#1}}{\textsubscript{#1}}}
% \def^{\let\@next\relax\ifmmode\@super\else\let\@next\@@super\fi\@next}
% \def_{\let\@next\relax\ifmmode\@sub\else\let\@next\@@sub\fi\@next}

\makeatother

\newcommand{\reppart}[1]{[: #1 :]}
\newcommand{\pick}[1]{\musicnote{\sffamily\MakeUppercase{#1}\normalfont}}
\newcommand{\num}{\beginverse}
\newcommand{\fin}{\endverse}
\newcommand{\start}[1]{\setcounter{oldversenum}{\value{versenum}}\setcounter{versenum}{#1}\beginverse}
\newcommand{\cl}{\endverse\setcounter{versenum}{\value{oldversenum}}}
\newcommand{\repsec}[2]{\start{#1} #2\\ \cl}
\newcommand{\emptyv}{\start{\nempty}}
%\newcommand{\emptyv}{\vspace{-\versesep}\start{\nempty}}
\newcommand{\freev}{\start{\nempty}}
\newcommand{\emptyspace}{\hspace{1pt}}
\newcommand{\chor}{\start{\nchorus}}
\newcommand{\intro}{\start{\nintro}}
\newcommand{\solo}{\start{\nsolo}}
\newcommand{\bridge}{\start{\nbridge}}
\newcommand{\chorusi}{\start{\nchorusi}}
\newcommand{\chorusii}{\start{\nchorusii}}
\newcommand{\averse}{\start{\naverse}}
\newcommand{\bverse}{\start{\nbverse}}
\newcommand{\cverse}{\start{\ncverse}}
\newcommand{\recite}{\start{\nrecite}}
\newcommand{\repchorus}[1]{\repsec{\nchorus}{#1}}
\newcommand{\repchorusi}[1]{\repsec{\nchorusi}{#1}}
\newcommand{\repchorusii}[1]{\repsec{\nchorusii}{#1}}
\newcommand{\chords}{\beginverse*}
% \newcommand{\repchorusadd}[1]{\beginchorus #1\\ \endchorus}
\newcommand{\cseq}[1]{\vspace{-\versesep}{\nolyrics #1}}
\newcommand{\ctab}[3]{\flushright\gtab{#1}{#2}{#3}}
\newcommand{\hidx}[1]{\textsuperscript{#1}}
\newcommand{\didx}[1]{\textsubscript{#1}}
\renewcommand{\rep}[1]{\sffamily\footnotesize($\times$#1)\normalsize\normalfont}
% \newcommand{\hidx}[1]{$^{\mathrm{#1}}$}
% \newcommand{\didx}[1]{$_{\mathrm{#1}}$}
% \authsepword{ a }

% \usepackage[pdftex]{hyperref}
% \hypersetup{colorlinks=false}

\pagenumbering{arabic}

\begin{document}
\sffamily
~\vspace{\stretch{1}}
\begin{center}
\Huge{}\textbf{KalaZpěvník}\normalsize\\[5ex]
2022\\[1ex]

\normalfont
\end{center}
\vspace{\stretch{2}}
\rmfamily
\newpage
\showindex[2]{Písně podle názvu}{titleidx}
\newpage
\begin{songs}{}
\beginsong{Alison Gross}[by={Asonance}]\hypertarget{song-1}{}\label{song-1}
\num
\[Em]{Když zapadlo} \[D]slunce a \[G]vkradla se \[H\hidx{7}]noc
\[Em]{A v šedivých} \[D]mracích se \[G]ztrá\[Am]cel \[H\hidx{7}]den
\[Em]{A když} síly \[D]zla ve tmě \[G]převzaly \[H\hidx{7}]moc
\[Em]{Tu Alison} \[D]Gross vyšla z hradu \[Em]{ven}
\fin
\chordsoff
\num
Tiše se vplížila na můj dvůr
a jak oknem mým na mě pohlédla
tak jen kývla prstem a já musel jít
a do komnat svých si mě odvedla
\fin
\chor
\chordson
\[Em]Alison Gross a černý \[D]hrad
\[G]{ze zlověstných} \[D]skal jeho hradby \[Em]ční
Alison Gross, nejodporněj\[D]ší
ze všech \[G]čaroděj\[H\hidx{7}]{nic v} zemi seve\[A]rní \[H\hidx{7}]
\cl
\emptyv
Složila mou hlavu na svůj klín
A sladkého vína mi dala pít
Já můžu ti slávu i bohatství dát
Jen kdybys mě chtěl za milenku mít
\cl
\bridge
Mlč a zmiz, babo odporná
Slepý jak krtek bych musel být
To radši bych na špalek hlavu chtěl dát
\ldots{} Než Alison Gross za milenku mít
\cl
\repchorus{\emptyspace}
\freev
Přinesla plášť celý z hedvábí
zlatem a stříbrem se celý skvěl
Kdybys jen chtěl se mým milencem stát
tak dostal bys vše, co bys jenom chtěl
\cl
\freev
Pak přinesla nádherný zlatý džbán
bílými perlami zářící
Kdybys jen chtěl se mým milencem stát
těch darů bys měl plnou truhlici
\cl
\bridge
Stůj a mlč, babo odporná
\ldots{} Než milencem tvým na chvíli se stát
\cl
\repchorus{\emptyspace}
\emptyv
Tu k ohyzdným rtům zvedla černý roh
a natřikrát na ten roh troubila
a s každým tím tónem mě ubylo sil
až všechnu mou sílu mi sebrala
\cl
\emptyv
Pak Alison Gross vzala čarovnou hůl
a nad mojí hlavou s ní kroužila
a podivná slova si zamumlala
a v slizkého hada mě zaklela
\cl
\repchorus{\emptyspace}
\emptyv
Tak uplynul rok a uplynul den
a předvečer svátku Všech svatých byl
a tehdy na místě, kde žil jsem jak had
se zjevila královna lesních víl
\cl
\emptyv
Dotkla se mě třikrát rukou svou
a její hlas kletbu rozrazil
a tak mi zas vrátila podobu mou
že už jsem se dál v prachu neplazil
\cl
\repchorus{\emptyspace}
\endsong

\beginsong{Batalion}[by={Spirituál kvintet}]\hypertarget{song-2}{}\label{song-2}
\bridge
\[Em]Víno \[G]máš a \[D]marky\[Em]tánku, \[G]dlouhá noc \[D]se \[Em]pro\[Hm]hý\[Em]ří
Víno \[G]máš a \[D]chvilku \[Em]spánku, \[G]díky, dí\[D]ky, \[Em]ver\[Hm]bí\[Em]ři
\cl
\chordsoff
\num
\chordson
\[Em]Dříve, než se rozední, kapitán \[G]{k osedlání} \[D]rozkaz \[Em]dá\[Hm]vá
\[Em]Ostruhami do slabin \[G]ko\[D]ně \[Em]po\[Hm]há\[Em]ní
Tam na straně polední čekají \[G]ženy, zlaťá\[D]{ky a} \[Em]slá\[Hm]{va}
\[Em]{Do výstřelů} karabin \[G]zvon \[D]už \[Em]vy\[Hm]zvá\[Em]{ní}
\fin
\chor
\chordson
\[Em]Víno na ku\[G]ráž a \[D]pomilovat marky\[Em]tánku
Zítra do Bur\[G]gund bata\[D]lion \[Em]za\[Hm]mí\[Em]ří
Víno na ku\[G]ráž a \[D]{k ránu} dvě hodiny \[Em]spánku
Díky, díky \[G]vám, králov\[D]ští \[Em]ver\[Hm]bí\[Em]ři
\cl
\num
Rozprášen je batalion, poslední vojáci se k zemi hroutí
Na polštáři z kopretin budou věčně spát
Neplač, sladká Marion, verbíři nové chlapce přivedou ti
Za královský hermelín padne každý rád
\fin
\repchorus{\emptyspace}
\bridge\emptyspace\\ \cl
\endsong

\beginsong{Bella Ciao}[by={Lidová}]\hypertarget{song-3}{}\label{song-3}
\num
\[Am]{Una mattina} mi son svegliato,
\fin
\chordsoff
\freev
O bella, ciao! Bella, ciao!
\chordson
Bella, c\[Am\hidx{7}]{iao, ciao,} ciao!
Una mat\[Dm]tina mi son svegl\[Am]iato
e ho tro\[E\hidx{7}]vato l'invas\[Am]or.
\cl
\num
O partigiano, portami via,
O bella, ciao! Bella, ciao!
Bella, ciao, ciao, ciao!
O partigiano, portami via,
ché mi sento di morir.
\fin
\num
E se io muoio da partigiano,
O bella, ciao! Bella, ciao!
Bella, ciao, ciao, ciao!
E se io muoio da partigiano,
tu mi devi seppellir.
\fin
\num
Seppellire lassù in montagna,
O bella, ciao! Bella, ciao!
Bella, ciao, ciao, ciao!
E seppellire lassù in montagna
Sotto l'ombra di un bel fior.
\fin
\num
E le genti che passeranno
O bella, ciao! Bella, ciao!
Bella, ciao, ciao, ciao!
E le genti che passeranno
Ti diranno «Che bel fior!»
\fin
\num
«È questo il fiore del partigiano»,
O bella, ciao! Bella, ciao!
Bella, ciao, ciao, ciao!
«È questo il fiore del partigiano
morto per la libertà!»
\fin
\endsong

\beginsong{Cestou do Jenkovic}[by={Radůza}]\hypertarget{song-4}{}\label{song-4}
\musicnote{akordy na konci sloky}
\intro
\cseq{\[Hm] \[A] \[G] \[A]}
\cl
\num
\[D]Můj děda z kola \[Hm]seskočil \[C]před prázdnou kašnou \[B]{na ná}\[C]{městí}
\[D]{Na lavičce} chleba \[Hm]posvačil, \[C]seřídil hodinky \[B]{na zá}\[C]{pěstí}
\fin
\chordsoff
\chor
\chordson
\reppart{A \[D]čápi z komína \[A]{od cihelny}
\[C]Zobákem klapou, asi jsou \[G]nesmrtelný}
\cl
\num
Tři kluci v bílejch košilích dělili se o poslední spartu
Ze zídky do záhonu skočili, přeběhli ulici a zmizeli v parku
\fin
\repchorus{\emptyspace}
\num
V oknech svítěj peřiny, na bílý kafe mlíko se vaří
Teď právě začaly prázdniny, venku je teplo a všechno se daří
\fin
\chor
\fadeout
\cl
\endsong

\beginsong{Čarodějnice z Amesbury}[by={Asonance}]\hypertarget{song-5}{}\label{song-5}
\num
Zuzana \[Em]byla dívka, \[D]která žila v \[Em]Amesbury
S jasnýma \[G]očima a \[D]řečmi pánům \[Em]navzdory
Souse\[G]dé o ní \[D]říkali, že \[Em]temná kouzla \[Hm]zná
A \[C]že se lidem \[Hm]vyhýbá a s \[C]ďáblem \[D]pletky \[Em]má
\fin\chordsoff\num
Onoho léta náhle mor dobytek zachvátil
A pověrčivý lid se na pastora obrátil
Že znají tu moc nečistou, jež krávy zabíjí
A odkud ta moc pochází, to každý dobře ví
\fin\num
Tak Zuzanu hned před tribunál předvést nechali
A když ji vedli městem, všichni kolem volali:
\uv{Už konec je s tvým řáděním, už nám neuškodíš,
teď na své cestě poslední do pekla poletíš!}
\fin\num
Dosvědčil jeden sedlák, že zná její umění
Ďábelským kouzlem prý se v netopýra promění
A v noci nad krajinou létává pod černou oblohou
Sedlákům krávy zabíjí tou mocí čarovnou
\fin\num
Jiný zas na kříž přísahal, že její kouzla zná
V noci se v černou kočku mění dívka líbezná
Je třeba jednou provždy ukončit ďábelské řádění
\chordson
A všichni křičeli jako posedlí: \uv{Na šibenici \[A]{s ní}!}
\fin\num
Spektrální důkazy pečlivě byly zváženy
Pak z tribunálu povstal starý soudce vážený
\uv{Je přece v knize psáno: nenecháš čarodějnici žít
a před ďáblovým učením budeš se na pozoru mít!}
\fin\num
Zuzana stála krásná s hlavou hrdě vztyčenou
A její slova zněla klenbou s tichou ozvěnou:
\uv{Pohrdám vámi, neznáte nic nežli samou lež a klam,
\chordson
pro tvrdost vašich srdcí jen, jen pro ni umí\[A]rám!}
\fin\num
Tak vzali Zuzanu na kopec pod šibenici
A všude kolem ní se sběhly davy běsnící
Ona stála bezbranná, však s hlavou vztyčenou
Zemřela tiše samotná pod letní oblohou
\fin
\endsong

\beginsong{Ďáblovy námluvy}[by={Asonance}]\hypertarget{song-6}{}\label{song-6}
\num
Svou \[Dm]krásnou \[C]píšťalu ti\[Dm]{ dám,}
hraje \[B]devět tónů\[C] na devět stran,
když \[Gm]půjdeš se mnou, \[Dm]lásko má
a \[B]když mě budeš\[Am]{ chtít.}
\fin
\chordsoff
\freev
Jen nech si tu píšťaličku sám,
hraj devět tónů na devět stran,
\chordson
já \[F]nejdu s tebou, \[C]lásko má,
já \[B]nechc\[C]{i s} tebo\[Dm]{u jít.}
\cl
\num
Já dám ti stuhy to vlasů,
mají devět barev pro krásu,
když půjdeš se mnou, lásko má,
a když mě budeš chtít.
\fin
\freev
Já nechci stuhy do vlasů,
mají devět barev pro krásu,
a nejdu s tebou, lásko má,
já nechci s tebou jít.
\cl
\num
Dám krásné šaty z hedvábí,
které devět krajek ozdobí,
když půjdeš se mnou, lásko má
a když mě budeš chtít.
\fin
\freev
Já nechci šaty z hedvábí,
které devět krajek ozdobí,
a nejdu s tebou, lásko má,
já nechci s tebou jít.
\cl
\num
Chceš devět černých perel mít
a na svatbu se vystrojit?
To všechno dám ti, lásko má,
jen když mě budeš chtít.
\fin
\emptyv
Já černé perly nechci mít,
a na svatbu se vystrojit,
a nejdu s tebou, lásko má,
já nechci s tebou jít.
\cl
\num
Já truhlu plnou zlata mám,
tu do vínku ti celou dám,
když půjdeš se mnou, lásko má,
a když mě budeš chtít.
\fin
\freev
Tvá slova příjemně mi zní,
tak připrav kočár svatební,
já půjdu s tebou, lásko má,
až kam jen budeš chtít.
\cl
\num
Tak ušli spolu devět mil,
když nohu v kopyto proměnil
a bledá dívka naříká,
už nechci s tebou jít.
\fin
\freev
Má milá, už tě nepustím,
zpět duši tvou ti nevrátím,
za trochu zlata, lásko má,
teď navždy budeš má.
\cl
\num
A jak tmou klopýtali dál,
vítr její smutnou píseň vál.
\chordson
Co \[Gm]dělat mám, já \[Dm]nešťastná
ach \[B]{co jen} dělat\[C] mám?
Co \[F]dělat mám, já \[C]nešťastná
ach \[B]{co jen}\[C] dělat\[Dm]{ mám?}
\fin
\endsong

\beginsong{Drunken Sailor}[by={Lidová}]\hypertarget{song-7}{}\label{song-7}
\num
\[Am]{What shall} we do with a drunken sailor
wh\[G]{at shall} we do with a drunken sailor
wh\[Am]{at shall} we do with a drunken sailor
early \[Em]{in the} m\[Am]orning?
\fin
\chordsoff
\chor
\chordson
W\[Am]{ay hay} and up she rises, W\[G]{ay hay} and up she rises
W\[Am]{ay hay} and up she rises early \[Em]{in the} m\[Am]orning.
\cl
\num
Shave his belly with a rusty razor
\fin
\num
Put him in a longboat till he's sober
\fin
\num
Stick him in a scupper with a hosepipe bottom
\fin
\num
Put him in the bed with the captain's daughter
\fin
\num
That's what we'll do with a drunken sailor.
\fin
\endsong

\beginsong{Dva havrani}[by={Asonance}]\hypertarget{song-8}{}\label{song-8}
\num
\chordson
Když jsem se \[Dm]{z pole} \[C]vrace\[Dm]la
Dva havrany jsem \[C]slyše\[Dm]la
Jak jeden \[F]druhého se \[Dm]ptá\[C]á:
\reppart{\uv{\[Dm]Kdo dneska veče\[C]{ři nám} \[Dm]dá?}}
\chordsoff
\fin\ifchorded\chordsoff\fi
\num
Ten první k druhému se otočil
A černým křídlem cestu naznačil
Krhavým zrakem k lesu hleděl
\reppart{A takto jemu odpověděl:}
\fin\ifchorded\chordsoff\fi
\num
Za starým náspem v trávě schoulený
Tam leží rytíř v boji raněný
A nikdo neví, že umírá
\reppart{Jen jeho kůň a jeho milá}
\fin\ifchorded\chordsoff\fi
\num
Jeho kůň dávno po lesích běhá
A jeho milá už jiného má
Už pro nás bude dosti místa
\reppart{Hostina naše už se chystá}
\fin\ifchorded\chordsoff\fi
\num
Na jeho bílé tváře usednem
A jeho modré oči vyklovem
A až se masa nasytíme
\reppart{Z vlasů si hnízdo postavíme!}
\fin\ifchorded\chordsoff\fi
\endsong

\beginsong{Já chci taky bejt skaut}[by={Budějovická Osmnáctka}]\hypertarget{song-9}{}\label{song-9}
\num
\[E]Můj život s životem se nepotkal
\[A]{A když} přemejšlím o tom, jak rvát se s ním dál
\[D]Říkám si, co bych v něm neudělal teď pořádnej \[E]říz
\chordsoff
Já chci mít velkej stan, kam se zmáčkne i slon
Pak taky švejcarák, co klidně přeřízne strom
Co otevře i flašky s pitím, který maj nehoráznej říz
\fin
\chordsoff
\num
Já chci mít klóbrc, jak maj lidi na klondajku
Taky kompas, co mi najde tu nejbližší knajpu
A sichrhajcku, když si v nouzi chci udělat tábornickej špíz
Chci poznat všechny barvy všech těch značenejch cest
A taky houby, ze kterejch se můžu sject
A tu tam taky něco navíc když pudu hrát třeba á zet kvíz
\fin
\bridge
\chordson
Všem svým \[G]neřestem teď vyhlásím boj
Vezmu si \[A]skautskou zbroj, co se jí říká kroj
\cl
\chorusi
\chordson
\[E]{Já chci} bejt velkej skaut
Bejt \[G]tak trochu in a tak trochu out
Bejt \[A]připravenej na každej vzlet i pád
Drát se \[C]křovím namísto \[D]cestou stád
\cl
\cverse
\chordson
\[E]{Ať se} říká, že je život pes
Ale když \[G]bez sirek zapálíš hned celej les
To pak \[A]život hnedle začne bejt velkej raut
Když seš \[C]tak trochu in a \[D]tak trochu out
\reppart{\[G]Hej, \[A]hej, já chci taky bejt \[E]skaut}
\cl
\num
Vopéct si buřta spálenýho jako ďábel
Umět zkracovačku když si musíš zkrátit kábel
A místo fejsu používat výhradně jen morseovo kód
Místo národních menšin lovit bobry
Mít za bráchy a ségry zlý i ty dobrý
Spát ve spacáku všude, kde to přijde jen malinkato vhod
\fin
\bridge
Všem svým neřestem teď hlásím boj
Beru si skautskou zbroj, co se jí říká kroj
\cl
\repchorusi{\emptyspace}
\chorusii
Lepší je civět dál než se dívat zpět
Na jedný noze přeskotačit celej svět
Je mnohem lepší než oběma v místě stát
I když se ti škarohlídi budou smát
\cl
\bverse
\chordson
Když \[C]pomůžeš babče vynést plný smetí
To \[G]chlapům spadne čelist, ženský se na tě sletí
Páč \[C]tohle to je život, tam nemůžeš zmáčknout reset
To \[A]nedělám si vůbec p****! \emph{(Dej si deset.)}
\cl
\repchorusi{\emptyspace}
\chorusii
\ldots{} i když se ti půlka světa bude smát \ldots{}
\cl
\cverse
Ať se říká, že je život pes...
\cl
\endsong

\beginsong{Jana smutná}[by={Jananas}]\hypertarget{song-10}{}\label{song-10}
\capo{2}
\musicnote{mezihra má akordy jako refrén}
\num
Je \[Em]smutné, když štěňátko na nádor umírá,
\chordsoff
Když prášek na spaní až ráno zabírá,
\chordson
\[A]Když raněná srna \[C]padá do mlází,
Když \[Em]starou paní záchranka před \[A]domem pora\[C]zí.
\fin
\chordsoff
\num
Je tuze smutné, když se bílý kůň stane lidskou potravou,
Když manželství zabíjí duši toulavou,
Když Waka-Waka-é-é je textem písňovým,
Když hygienik otráví se sýrem plísňovým.
\fin
\chor
\chordson
\[Em]{Já vím,} \[D]každý \[C]trápení své \[Em]má
Ale \[Am]nejsmutnější \[D]{na světě} jsem \[Em]{já}
\cl
\bridge\emptyspace\\ \cl
\num
Je smutné, když ti bratr řekne, že jsi jeho syn,
a když se po neštovicích fotíš do novin,
když půlku českých obrazů odváží si Švéd,
a když tě tvůj kluk z Plzně už vůbec nemá réd.
\fin
\num
Je smutné, když se na konečné ráno probudíš,
a v podstatě vzato lituješ, že bdíš,
když slabý, křehký racek zmírá pod naftou,
když na Slovensku přestane se blýskat nad Tatrou.
\fin
\chor
\rep{2}
\cl
\bridge\emptyspace\\ \cl
\num
Je smutné, když se banjo večer naladí,
a ty zjistíš, že už ti to vlastně nevadí,
když tě kamarádi na párty nepozvou,
a ani nepřijdou, když máš párty svou.
\fin
\num
Když fotbalista modelce dá náhle kopačky,
a modelka pak před přehlídkou snídá kapačky,
a s Karlem Gottem naopak když nikdo nesnídá,
tak to se smutné zdá, ale nejvíc trpím já.
\fin
\chor
\rep{2}
\ldots{}
Ať se nikdo nezlobí, smůla má se násobí
Já mám prostě těžší období
Ať se nikdo nediví, svět je nespravedlivý
Já jsem ten, kdo tohle nejlíp ví
Já vím, každý trápení své má
Ale nejsmutnější na světě jsem já!
Já vím, každý trápení své má
Ale nejsmutnější na světě jsem já!
\cl
\endsong

\beginsong{Jdem zpátky do lesů}[by={Pavel Lohonka Žalman}]\hypertarget{song-11}{}\label{song-11}
\num
\[Am\hidx{7}]Sedím na kolejích, \[D\hidx{7}]které nikam neve\[G]dou, \[C] \[G]
\[Am\hidx{7}]koukám na kopretinu, jak \[D\hidx{7}]miluje se s lebe\[G]dou. \[C] \[G]
\[Am\hidx{7}]Mraky vzaly slunce \[D\hidx{7}]zase pod svou ochra\[G]nu, \[Em]
\[Am\hidx{7}]{jen ty} nejdeš, holka zlatá, \[D\hidx{7}]kdypak já tě dosta\[G]nu? \[D\hidx{7}]
\fin\chor
\[G]{Z ráje}, my vyhnaní \[Em]{z ráje},
kde není už \[Am\hidx{7}]místa, \[C]prej něco se \[G]chystá, \[D]óóó,
\[G]{z ráje} nablýskaných \[Em]plesů
jdem zpátky do \[Am\hidx{7}]lesů~-- \[C]za nějaký \[G]čas.
\cl\chordsoff\num
Vlak nám včera ujel ze stanice do nebe,
málo jsi se snažil, málo šel jsi do sebe.
Šel jsi vlastní cestou, a to se dneska nenosí,
i pes, kterej chce přízeň, napřed svýho pána poprosí.
\fin
\repchorus{Z ráje \ldots}
\num
Už tě vidím z dálky, jak máváš na mě korunou,
jestli nám to bude stačit, zatleskáme na druhou.
Zabalíme všechny, co si dávaj rande za branou,
v ráji není místa, možná v pekle se nás zastanou.
\fin
\repchorus{Z ráje \ldots}
\endsong

\beginsong{Jelen}[by={Jelen}]\hypertarget{song-12}{}\label{song-12}
\num
\[Dm]{Na jaře} se vrací \[C]{od podzima} li\[Dm]stí
Mraky místo ptáků \[C]krouží nad Závi\[Dm]stí
Kdyby jsi se někdy \[C]{ke mně} chtěla vrá\[Dm]tit
Nesměla bys, lásko, \[C]moje srdce ztra\[G]tit
\fin
\chordsoff
\chor
\chordson
\[Dm]Zabil jsem v lese \[C]jele\[F]na
Bez nenávisti, \[C]bez jmé\[Dm]na
Když přišel dolů k \[C]řece \[F]pít
Krev teče do vody, \[C]{v srdci} \[Dm]klid
\cl
\emptyv
\chordson
\cseq{\[Dm] \[C] \[F] \[C]} \rep{2}
\cl
\num
Voda teče k moři, po kamenech skáče
Jednou hráze boří, jindy tiše pláče
Někdy mám ten pocit, i když roky plynou
Že vidím tvůj odraz dole pod hladinou
\fin
\repchorus{\emptyspace}
\num
Na jaře se vrací listí od podzima
Čas se někam ztrácí, brzo bude zima
Svět přikryje ticho, tečka za příběhem
Kdo pozná, čí kosti zapadaly sněhem
\fin
\repchorus{\emptyspace}
\emptyv
Hej!
\cl
\chor
\emph{(o půltón výše)}
\cl
\endsong

\beginsong{K Táboráčku}[by={Skautské písně}]\hypertarget{song-13}{}\label{song-13}
\num
\[G]Klidně čučte\[D] na novu a
\[Em]mějte rád\[C]{i coca} colu
\[G]{v Mc} Donaldu z\[D]dravě jezte
\[Em]{pro mě} za mě\[C] třeba zmizte
\fin
\chordsoff
\num
\chordson
\[G]{Do U} S A \[D]tam je skvěle
\[Em]{Jen jim} lezte\[C] na na na na
\[G]přeju vám a\[D]{ť se} vejdete
\[Em]Jedno mi prosím\[C] řekněte
\fin
\bridge
\chordson
\[D]Kdy konečně přestanete
\[C]{na můj} kroj se hloupě ptát
\[D]vlastně mě to nezajímá
\[C]{na vás} se můžem \[D]{Á dva} tři
\cl
\chor
\emph{(akordy jako sloka)}
/: \emph{(Jsem)} skaut jsem skaut
jsem hrdej na to že jsem OUT
že jsem skaut ještě neznamená
že pogo nemám rád:/
\cl
\num
Klidně sviťte fialově
metrák třpytek gel na hlavě
aspoň tunu i na chlupy
ať jste dobře osliznutý
\fin
\num
Říkáte že posloucháme
jen Žalmana, country, folk, pop
no to známe ale taky máme
vlastní styl- scout rock
\fin
\bridge
Dámy a pánové
bárbíny a kenové
jen si klepejte na čelo
kašleme vám na kenvelo!
\cl
\repchorus{\emptyspace}
\endsong

\beginsong{Lokomotiva}[by={Poletíme}]\hypertarget{song-14}{}\label{song-14}
\emptyv
\cseq{\[G] \[D] \[Em] \[C]}\\
\cl
\chordsoff
\num
Pokaždé když tě vidím, vím, že by to šlo
A když jsem přemejšlel, co cítím, tak mě napadlo
Jestli nechceš svýho osla vedle mýho osla hnát
Jestli nechceš se mnou tahat ze země rezavej drát
\fin
\chor
\chordson
\[G]Jsi loko\[D]motiva, kte\[Em]{rá se} řítí \[C]tmou
\[G]Jsi indi\[D]áni, kteří \[Em]prérií je\[C]dou
\[G]Jsi kulka \[D]vystřelená \[Em]{do mojí} hla\[C]vy
\[G]Jsi prezi\[D]dent a já tvé \[Em]Spojené stá\[C]ty
\cl
\num
Přines jsem ti kytku, no co koukáš, to se má
Je to koruna žvejkačkou ke špejli přilepená
A dva kelímky od jogurtu, co je mezi nima niť
Můžeme si takhle vždycky volat, když budeme chtít
\fin
\repchorus{\emptyspace}
\num
Každej příběh má svůj konec, ale né ten náš
Nám to bude navždy dojit, všude kam se podíváš
Naše kachny budou zlato nosit a krmit se popcornem
Já to každej večer spláchnu půlnočním expresem
\fin
\repchorus{\emptyspace}
\num
Dětem dáme jména Jessie, Jeddej, Jad a John,
Ve stopadesáti letech ho budu mít stále jako slon
A ty neztratíš svoji krásu, stále štíhlá kolem pasu
Stále dokážeš mě chytit lasem a přitáhnout na terasu
\fin
\chor
\rep{2}
\cl
\endsong

\beginsong{Marie}[by={Tomáš Klus}]\hypertarget{song-15}{}\label{song-15}
\emptyv
\cseq{\[C] \[E] \[F] \[G]}
\cl\chordsoff\num
Je den~-- tak pojď, Marie, ven
Budeme žít~-- a házet šutry do oken
Je dva necháme doma trucovat
Když nechtějí, nemusí~-- nebudem se vnucovat
Jémine~-- všechno zlý jednou pomine
Tak Marie~-- co ti je?
\fin\num
Všemocné jsou loutkařovy prsty
Ať jsou tenký nebo tlustý, občas přetrhají nit
A to pak jít~-- a nemít nad sebou svý jistý
Pořád s tváří optimisty listy v žití obracet
Je to jed~-- mazat si kolem huby med
A neslyšet, jak se ti bortí svět
Marie~-- kdo přežívá, nežije, tak ádijé
\fin\num
Marie~-- už zase máš k tulení sklony
Jako loni slyším kostelní zvony znít
A to mě zabije, a to mě zabije
A to mě zabije~-- jistojistě!
\fin\chor
Já mám, Marie, rád, když má moje bytí spád
Býti věčně na cestách
A k ránu spícím plícím život vdechovat
Nechtěj mě milovat
Nechtěj mě milovat
Nechtěj mě milovat!
\cl\num
Copak nemůže být mezi ženou a mužem
Přátelství, kde není nikdo nic dlužen
Prostě jen prosté spříznění duší
Aniž by kdokoli cokoli tušil
Na na na\ldots
\fin
\repchorus{\emptyspace}
\endsong

\beginsong{Mezi horami}[by={Čechomor}]\hypertarget{song-16}{}\label{song-16}
\num
\reppart{\[Am]Mezi \[G]hora\[Am]mi \[C]lipka \[G]zele\[C]ná}
\reppart{\[C]Zabili Janka, \[G]Janíčka, \[Am]Janka, \[Am]miesto \[Em]jele\[Am]ňa}
\fin\chordsoff\num
\reppart{Keď ho zabili, zamordovali}
\reppart{Na jeho hrobě, na jeho hrobě kříž postavili}
\fin\num
\reppart{Ej, křížu, křížu, ukřižovaný}
\reppart{Zde leží Janík, Janíček, Janík, zamordovaný}
\fin\ifchorded\chordsoff\fi
\num
\reppart{Tu šla Anička plakat Janíčka}
\reppart{Aj, na hrob padla a viac nevstala~-- dobrá Anička}
\fin\ifchorded\chordsoff\fi
\endsong

\beginsong{Na co nesmíš zapomenout}[by={MIDI Lidi}]\hypertarget{song-17}{}\label{song-17}
\intro
\cseq{\[E\flt{}] \[G\shrp{}] \[G\shrp{}] \[E\flt{}]}\\
\cl
\chordsoff
\num
\chordson
\[E\flt{}]{Tak jsem} si přišel s tebou \[G\shrp{}]popovídat.
\[G\shrp{}]{A rady} ti dát \[E\flt{}]{a rady} ti dát.
\fin
\num
Na světe je dobře jenom když ho máš rád,
ale kde to mám brát? Ale kde to mám brát?
\fin
\num
Nemusíš už nikdy nechat na sebe řvát,
tak do oka drát, tak do oka drát.
\fin
\num
Že seš borec nemusíš už na sebe hrát.
Trapný chvíle nejsou s tím, kdo tě má rád.
\fin
\num
Nemusí ti překážet prašivej stát,
můžeš občas taky něco do rukou brát.
\fin
\num
Není nutný jenom doufat, že to má řád,
můžeš ho ty tomu dát. Můžeš ho ty tomu dát.
\fin
\chor
\emph{(pískání)}
\cl
\num
A jestli už máš zas náladu poraženou,
jestli si tvá hlava vzala dovolenou.
A když se cítíš zas jak kafka před proměnou,
tak ti zkusím připomenout.
\fin
\num
A nemusíš se nikdy plazit po kolenou.
Nenechej si prací zkazit dovolenou.
A není nutný vždycky stát na červenou,
ale vetšinou choď na zelenou.
\fin
\num
A nemusíš se vždycky jenom holit pěnou.
Krize bývá rukou trhu prodlouženou.
A jestli si chtěl večer strávit se svou ženou,
tak na to nesmíš zapomenout.
\fin
\emptyv
\chordson
\cseq{\[E\flt{}] \[G\shrp{}]}\\
\cl
\freev
\chordson
\[G\shrp{}]{Tak na} to nesmíš zapomenout.\[E\flt{}]
\cl
\emptyv
\chordson
\cseq{\[E\flt{}] \[G\shrp{}]}\\
\cl
\freev
\chordson
\[G\shrp{}]{Tak na} to nesmíš zapomenout.\[E\flt{}]
\cl
\freev
\emph{(pískání)}
\cl
\endsong

\beginsong{Nagasaki Hirošima}[by={Mňága a Žďorp}]\hypertarget{song-18}{}\label{song-18}
\num
\[G]Tramvají \[D]dvojkou \[C]jezdíval \[D]jsem do Žide\[G]nic \[D] \[C] \[D]
\[G]Z takový \[D]lásky \[C]většinou \[D]nezbyde \[Em]nic
\[C]Z takový \[G]lásky \[C]jsou kruhy \[G]pod oči\[D]ma
A \[G]dvě spálený \[D]srdce~-- \[C]Nagasaki \[D]Hiroši\[G]ma \[D] \[C] \[D]
\fin\chordsoff\num
Jsou jistý věci, co bych tesal do kamene
Tam, kde je láska, tam je všechno dovolené
A tam kde není, tam mě to nezajímá
Jó, dvě spálený srdce~-- Nagasaki Hirošima
\fin\num
Já nejsem svatej~-- ani ty nejsi svatá
Jablka z ráje bejvala jedovatá
Jenže hezky jsi hřála, když mi někdy byla zima
Jó, dvě spálený srdce~-- Nagasaki Hirošima
\fin
\repsec{1}{\ldots\\
\reppart{A dvě spálený srdce~-- Nagasaki Hirošima} \rep{3}}
\endsong

\beginsong{Nikdy nic nebylo}[by={Sto zvířat}]\hypertarget{song-19}{}\label{song-19}
\emptyv
\cseq{\[Am] \[F] \[D] \[G]}
\cl\chordsoff\num
Nikdy jsi nebyla a naše seznámení
Proběhlo má milá před kinem, který není
Nepil jsem Tequilu~-- a ne že bych se šklebil
Netlačil na pilu v tom baru, kterej nebyl
\fin\num
Nikdy jsem neříkal, kolik mě čeká slávy
Nehrála muzika, ze který jsem se dávil
Nechtěl jsem na závěr Ti vyblejt celej život
A ztratit charakter a všechno oběživo
\fin\chordson\emptyv
\cseq{\[Am] \[C] \[D] \[Am]}
\cl\chordsoff\chor
Nebylo nic~-- já jen, kdybys měla chvíli
můžem si někam sednout
Nebylo nic~-- pár let jsme spolu nemluvili
a předtím taky ani jednou
\cl\num
Nikdy nic nebylo~-- noc jako horská dráha
Vsadím se o kilo, že prostě nejsi drahá
Je to jen chiméra~-- a žádný že jsem brečel
a měl jsem hysterák, ty už mi neutečeš
\fin\num
Scénář se nekoná~-- my dva ho nenapsali
a někdo místo nás prázdnej papír spálil
Nic není ani já, ani tvý zlatý oči
Jen jsme šli na biják, co nikdo nenatočil
\fin
\solo
\emptyspace
\cl
\repchorus{\emptyspace}
\num
Nikdy jsi nebyla a naše seznámení
Proběhlo má milá před kinem, který není
Nic není ani já, ani tvý zlatý oči
Jen jsme šli na biják, co nikdo nenatočil
\fin
\repchorus{\fadeout}
\endsong

\beginsong{Omnia vincit Amor}[by={Klíč}]\hypertarget{song-20}{}\label{song-20}
\num
\[Dm]{Šel pocestný} kol \[C]hospodských \[Dm]zdí
\[F]Přisedl k nám a \[C]lokálem \[F]zní
\[Gm]Pozdrav jak svaté \[Dm]přikázá\[C]ní:
\[Dm]Omnia \[C]vincit \[Dm]Amor
\fin
\chordsoff
\num
Hej, šenkýři, dej plný džbán
Ať chasa ví, kdo k stolu je zván
Se mnou ať zpívá, kdo za své vzal
Omnia vincit Amor
\fin
\chor
\chordson
Zlaťák \[F]pálí, \[C]nesleví \[Dm]nic
Štěstí v \[F]lásce \[C]znamená \[F]víc
Všechny \[Gm]pány \[F]{ať vezme} \[C]ďas!
\[Dm]Omnia \[C]vincit \[Dm]Amor
\cl
\num
Já viděl zemi válkou se chvět
Musel se bít a nenávidět
V plamenech pálit prosby a pláč
Omnia vincit Amor
\fin
\num
Zlý trubky troubí, vítězí zášť
Nad lidskou láskou roztáhli plášť
Vtom kdosi krví napsal ten vzkaz
Omnia vincit Amor
\fin
\num
Já prošel každou z nejdelších cest
Všude se ptal, co značí ta zvěst
Až řekl moudrý: \uv{Pochopíš sám!}
Omnia vincit Amor
\fin
\num
Teď s novou vírou obcházím svět
Má hlava zšedla pod tíhou let
Každého zdravím tou větou všech vět
[: Omnia vincit Amor :] \rep{3}  \fadeout
\fin
\endsong

\beginsong{Proměny}[by={Čechomor}]\hypertarget{song-21}{}\label{song-21}
\num
\[Am]Darmo sa ty trápíš \[G]můj milý sy\[C]nečku
Nenosím já tebe, \[E]nenosím v sr\[Am]déčku
A já tvo\[G]ja \[C]ne\[G]bu\[C]du \[Dm]ani jednu \[E]hodi\[Am]nu
\fin\chordsoff\num
Copak sobě myslíš má milá panenko
Vždyť ty jsi to moje rozmilé srdénko
A ty musíš býti má, lebo mi tě Pán Bůh dá
\fin\num
A já sa udělám malú veveričkú
A já ti uskočím z dubu na jedličku
Přece tvoja nebudu ani jednu hodinu
\fin\num
A já chovám doma takú sekérečku
Ona mi podetne dúbek i jedličku
A ty musíš býti má lebo mi tě Pán Bůh dá
\fin\num
A já sa udělám tú malú rybičkú
A já ti uplynu preč po Dunajíčku
Přece tvoja nebudu ani jednu hodinu
\fin\num
A já chovám doma takovú udičku
Co na ni ulovím kdejakú rybičku
A ty musíš býti má lebo mi tě Pán Bůh dá
\fin\num
A já sa udělám tú velikú vranú
A já ti uletím na uherskú stranu
Přece tvoja nebudu ani jednu hodinu
\fin\num
A já chovám doma starodávnú kušu
Co ona vystřelí všeckým vranám dušu
A ty musíš býti má lebo mi tě Pán Bůh dá
\fin\num
A já sa udělám hvězdičkú na nebi
A já budu lidem svítiti na nebi
Přece tvoja nebudu ani jednu hodinu
\fin\num
A sú u nás doma takoví hvězdáři
Co vypočítajú hvězdičky na nebi
A ty musíš býti má lebo mi tě Pán Bůh dá
\fin
\endsong

\beginsong{Ráda tohle, ráda tamto}[by={Karel Plíhal}]\hypertarget{song-22}{}\label{song-22}
\chor
\[Am]Ráda se miluje, \[G]ráda \[C]jí, \[F]ráda si \[Em]jenom tak \[Am]zpívá
Vrabci se na plotě \[G]háda\[C]jí, \[F]kolikže \[Em]času jí \[Am]zbývá
\cl\num
\[F]Než vítr dostrká \[C]k útesu \[F]tu její legrační \[C]bár\[E]ku
\[Am]Pámbu si ve svým \[G]note\[C]su \[F]udělá \[Em]jen další \[Am]čárku
\fin\chordsoff
\repchorus{\emptyspace}
\num
Psáno je v nebeské režii, a to hned na první stránce
že naše duše nás přežijí v jinačí tělesný schránce
\fin
\repchorus{\emptyspace}
\num
Úplně na konci paseky, tam, kde se ozvěna tříští
sedí šnek ve šneku pro šneky~-- snad její podoba příští
\fin
\repchorus{\emptyspace}
\endsong

\beginsong{Rosa na kolejích}[by={Wabi Daněk}]\hypertarget{song-23}{}\label{song-23}
\begin{textblock}{1.5}(10.5,0)\noindent \gtab{F\hidx{6}}{003231} \gtab{F\shrp{}\hidx{6}}{XX4342} \gtab{G\hidx{6}}{3:003231} \end{textblock}
\num
\[C]Tak jako jazyk \[F\hidx{6}]stále \[F\shrp{}\hidx{6}]nará\[G\hidx{6}]{ží na} vylomený \[C]zub,
tak se vracím \[F\hidx{6}]{k svýmu} \[F\shrp{}\hidx{6}]nádra\[G\hidx{6}]ží, abych šel zas \[C]dál.
Přede mnou \[F\hidx{6}]stíny se \[G\hidx{6}]plouží a \[Am]{nad krajinou} \[C\didx{dim}]krouží
podivnej pták, \[F\hidx{6}] \[F\shrp{}\hidx{6}] \[G\hidx{6}] pták nebo \[C]mrak.
\fin\chor
Tak do toho \[F\hidx{6}]šlápni, ať \[G\hidx{6}]vidíš kousek \[C]světa,
vzít do dlaně \[F\hidx{6}]dálku \[G\hidx{6}]zase jednou \[C]zkus,
telegrafní \[F\hidx{6}]dráty \[G\hidx{6}]hrajou ti už \[C]léta
to nekonečně \[F\hidx{6}]dlou\[F\shrp{}\hidx{6}]hý \[G\hidx{6}]mono\[F\shrp{}\hidx{6}]tón\[F\hidx{6}]ní \[C]blues.
Je ráno, je ráno! Nohama \[F\hidx{6}]stí\[F\shrp{}\hidx{6}]ráš \[G\hidx{6}]rosu na \[F\shrp{}\hidx{6}]ko\[F\hidx{6}]le\[C]jích.
\cl\chordsoff\num
Pajda dobře hlídá pocestný, co se nocí toulaj,
co si radši počkaj, až se stmí, a pak šlapou dál.
Po kolejích táhnou bosí a na špagátku nosí
celej svůj dům, deku a rum.
\fin
\repchorus{Tak do toho šlápni\ldots}
\cverse
Nohama stíráš rosu na kolejích.
\cl
\endsong

\beginsong{Sáro!}[by={Traband}]\hypertarget{song-24}{}\label{song-24}
\chorusi
\[Am]Sáro, \[Em]Sáro, \[F]{v noci} se mi \[C]zdálo
Že \[F]tři andělé \[C]Boží k nám \[F]přišli na o\[G]běd
\[Am]Sáro, \[Em]Sáro, jak \[F]moc a nebo \[C]málo
Mi \[F]chybí, abych \[C]tvojí duši \[F]mohl rozu\[G]mět?
\cl\chordsoff\num
Sbor kajícných mnichů jde krajinou v tichu
A pro všechnu lidskou pýchu má jen přezíravý smích
A z prohraných válek se vojska domů vrací
Však zbraně stále burácí a bitva zuří v nich
\fin\num
Vévoda v zámku čeká na balkóně
Až přivedou mu koně, pak mává na pozdrav
A srdcová dáma má v každé ruce růže
Tak snadno pohřbít může sto urozených hlav
\fin\num
Královnin šašek s pusou od povidel
Sbírá zbytky jídel a myslí na útěk
A v podzemí skrytí slepí alchymisté
Už objevili jistě proti povinnosti lék
\fin\chorusii
Sáro, Sáro, v noci se mi zdálo
Že tři andělé k nám přišli na oběd
Sáro, Sáro, jak moc a nebo málo
Ti chybí, abys mojí duši mohla rozumět?
\cl\num
Páv pod tvým oknem zpívá sotva procit
O tajemstvích noci ve tvých zahradách
A já~-- potulný kejklíř, co svázali mu ruce
Teď hraju o tvé srdce a chci mít tě na dosah
\fin\cverse
Sáro, Sáro, pomalu a líně
S hlavou na tvém klíně chci se probouzet
Sáro, Sáro, Sáro, Sáro, rosa padá ráno
A v poledne už možná bude jiný svět!
\chordson
\[F]Sáro, \[C]Sáro, \[F]vstávej, milá \[C]Sáro
\[F]Andělé k nám \[Dm]přišli na o\[C]běd
\chordsoff
\fin
\endsong

\beginsong{Sluneční Hrob}[by={Blue Effect}]\hypertarget{song-25}{}\label{song-25}
\freev
\chordsoff
Usínám a chtěl bych se vrátit o nějakej ten rok zpátky,
bejt zase malým klukem, kterej si rád hraje a který je s tebou.
\cl
\chordsoff
\num
\chordson
\[E]Zdá se \[F\shrp{}m]{mi,  } \[G\shrp{}m]{je to} moc \[F\shrp{}m]let,
\chordsoff
já byl kluk,  kterej chtěl
\chordson
\[E]znáti s\[F\shrp{}m]{vět, } \[G\shrp{}m]{s tebou} js\[F\shrp{}m]{em si} hr\[E]ál.
\fin
\num
\chordson
\[E]Vrátím \[F\shrp{}m]{se a c}\[G\shrp{}m]{htěl bych} \[F\shrp{}m]rád
\chordsoff
být s tebou, zavzpomínat,
\chordson
\[E]mám tu \[F\shrp{}m]{teď  } \[G\shrp{}m]{ale zpr}\[F\shrp{}m]{ávu zl}\[E]ou\[E\hidx{7}]{.}
\fin
\chor
\chordson
\[C\shrp{}m]Su-c\[E\flt{}m]{há hlí}\[A]na \[G\shrp{}m]ta~-- \[F\shrp{}m]dy,
\[C\shrp{}m]bez \[E\flt{}m]kví~-- \[A]tí, \[G\shrp{}m]{bez vo~}\[F\shrp{}m]{- dy,}
\[G\shrp{}m]{já na} n\[F\shrp{}m]{i poklekám,}
\[G\shrp{}m]vzpomínkou pocta\[H\hidx{7}]{ se vzdává.}
\cl
\num
Loučím se a něco však
tam zůstalo z těch našich dnů,
já teď vím, věrný zůstanu.
\fin
\chor
Suchá hlína tady, \ldots{}
\cl
\num
Loučím se a něco však
tam zůstalo z těch našich dnů,
já teď vím, věrný zůstanu.
\fin
\freev
Nemohu spát, probouzím se a zase se nemohu ubránit myšlence,
vrátit se o nějakej ten rok zpátky, bejt zase malým klukem,
který si rád hraje, který je s tebou.
\cl
\endsong

\beginsong{Studený nohy}[by={Radůza}]\hypertarget{song-26}{}\label{song-26}
\num
\[Em]Prší, \[Hm] \[Am]choulím se \[Hm]{do svrchníku}
\[Em]{Než se} \[Hm]otočím \[C]{na pod}\[D]patku
\chordsoff
Zalesknou se světla na chodníku
Jak pětka na věčnou oplátku
\fin
\chordsoff
\num
Slyším kroky zakletejch panen
To je vínem, to je ten pozdní sběr
Každá kosa najde svůj kámen
To je vínem, ber mě, ber
\fin
\chorusi
\chordson
Studený \[C]nohy \[H]schovám doma \[Em]{pod peřinou}
A ráno \[C]kafe dám si \[H]hustý jako \[Em]tér
Přežiju \[C]tuhle nedě\[H]{li tak} jako \[Em]každou jinou
\ldots{} Na koho \[C]slovo padne, \[H]ten je soli\[A]tér
\cl
\num
Broukám si píseň o klokočí
Prší a dlažba leskne se
Je chladno a hlava, ta se točí
Jak světla na plese
\fin
\chorusi
\rep{2}
\cl
\num
Tak mám a nebo nemám kliku
Zakletá panna směje se
A moje oči, lesknou se na chodníku
Jak světla na plese
\fin
\chorusii
\chordson
\ldots{} Na koho \[C]slovo padne, \[H]ten je soli\[Em]tér
\cl
\chorusii
\rep{2}
\cl
\endsong

\beginsong{Tak dej groš zaklínači}[by={Joey Batey}]\hypertarget{song-27}{}\label{song-27}
\musicnote{CAPO I.}
\num
\[Am]{  Jak} já skromný   \[Dm]bard
poctěn na sto  \[F] krát
táhl s Geraltem z  \[G] Rivie
chci \[E]{  do} písně   \[Am]dát
\fin
\chordsoff
\num
Když bílý vlk se pral
s Ďáblem stříbrným
jeho armádou elfů
nechal jít světem
\fin
\num
  Po mě skočili
a barda přešel smích
ztřískali loutnu
šli po zubech mých
\fin
\num
rohy ďáblovy
čtvrtili mi hřbet
a zaklínač zařval:
Nech ho a hned
\fin
\chor
\chordson
\[Am]   Tak dej \[E]{  groš} Zaklí   \[C]nači
no \[Dm]{  tak} nebuď  \[Am]{ skoupý}
\chordsoff
no tak nebuď skoupý oouou
\chordson
Tak dej \[E]{  groš} Zaklí\[C]nači
no \[Dm]{  tak} nebuď  \[E] skoupý
\cl
\num
  Až na kraj světa jde,
rve se s démony,
co nutí tě k pláči
a zarmoucení
\fin
\num
Elfy zahání
zpátky do ohrad,
do vzdálených kopců,
jak velí řád
\fin
\num
A ty neřáde
jen drtíš jeho hruď
on lidstvo jen chrání
tak hradbou mu buď
\fin
\num
to je příběh můj,
má být šampiónem zván
vždyť přemohl zlo,
tak naplň mu džbán
\fin
\chor
\rep{3}
\cl
\endsong

\beginsong{V Lese}[by={Pokáč}]\hypertarget{song-28}{}\label{song-28}
\intro
\cseq{\[F,] \[C] \[G] \[Am]}\\
\cl
\chordsoff
\num
\chordson
\[F]{  dneska} všichni  \[C] říkají a   vš\[G]ude píše   se   \[Am]
\chordsoff
že prej je moderní a zdravý jen tak chodit po lese
a já bych řek že na tom kruci asi něco musí bejt
když o tom zpívá na svým albu už i Justin Timberlake
\fin
\num
já do přírody od přírody vůbec nechodím
mám však spoustu přátel hipsterů ti mi jistě napoví
aha tak omlouvám se prý že tam též nikdy nebyli
však na cestu mi daj raw guláš a z flanelu košili
\fin
\bridge
\chordson
a tak \[F]{  šel} jsem tam však byl jsem zpátky hned
\[G]{  páč} tam maj fakt marnej internet
\cl
\chor
\chordson
\reppart{dneska \[F]{  jsem} v lese \[C]{  byl}
a už  \[G] tam nepu  \[Am]{ du}
\chordsoff
nikde jsem nezažil takhle ukrutnou nudu}
\cl
\num
vážně nevím co s tím lesem pořád všichni maj
jen jsem tam vlez hned jsem si zasvinil svý boty značky nike
žádnej signál žádná wifi žádný data zde nemám
teda data jsou tu ečko a to fakt nepočítám
\fin
\num
světlo pro selfíčka je zde ultra zoufalý
řek bych že kdyby to tu vykáceli líp by uďáli
ať postaví tu pro mě za mě třeba další obchoďák
tam moh bych shopovat a právě by mě nepotřísnil pták
\fin
\bridge
\chordson
samá  \[F] houba čerstvej   \[Am]vzduch a samej hmyz
do \[G]{  přírody} fakt chodit neměl bys
\cl
\chor
\reppart{Dneska jsem v lese byl \ldots{}}
\cl
\freev
\chordson
oo \[F]{  oo} \[C]{  ooo} jak se do lesa \[G]{  volá} tak se z něj ozý\[Am]{vá}
\chordsoff
oooooo tak poslouchej mě lese já mám radši obývák
oooooo proč vyrážet na místa lidstvem netknutý
ooooooo když tam nejsou žádný zásuvky a žádný bubble tea
\cl
\chorusii
Dneska jsem v lese byl
bylo to naposled
abych maily vyřídil
musel jsem vylízt na posed
\cl
\endsong

\beginsong{Vlaštovky}[by={Traband}]\hypertarget{song-29}{}\label{song-29}
\num
\[Dm]Každé jaro \[Am]{z velké} dáli \[B]vlaštovky k nám \[F]přilétaly
\[Dm]Někdy až \[C]dovnitř do stave\[Am]{ní}
\chordsoff
Pod střechou se uhnízdily a lidé, kteří uvnitř žili
Rozuměli jejich švitoření
\fin
\chordsoff
\num
O dalekých krajích, hlubokých mořích, divokých řekách
Vysokých horách, které je nutné přelétnout
O nebeských stezkách, zářících hvězdách, o cestách domů
O korunách stromů, kde je možné odpočinout
\fin
\num
Jsme z míst, která jsme zabydlili, z hnízd, která jsme opustili
Z cest, které končí na břehu
Jsme z lidí i z všech bytostí, jsme z krve, z masa, z kostí
Ze vzpomínek, snů i z příběhů
\fin
\num
Jsme jako ti ptáci, z papíru draci, létáme v mracích
A pak se vracíme zpátky tam, kde připoutaní jsme
Jsme lidské bytosti z masa a kostí, jsme jenom hosti
Na tomhle světě~-- přicházíme, odcházíme
\fin
\num
A chceme mít jisto, že někde místo, že někde je hnízdo
Odkud jsme přišli a kam zas potom půjdeme spát
Že někde je domov, že někde je hnízdo, útulno, čisto
Někde je někdo, kdo čeká na nás, na návrat
\fin
\num
Tam v dalekých krajích, v hlubokých mořích, v divokých řekách
Ve vysokých horách, které je nutné přelétnout
Tam v nebeských stezkách, v zářících hvězdách
Na cestách domů, v korunách stromů, kde je možné odpočinout
\fin
\repsec{6}{\emptyspace}
\endsong

\beginsong{Vzhůru na palubu}[by={Karel Černoch}]\hypertarget{song-30}{}\label{song-30}
\averse
\chordson
\[Am]Vzhůru na palubu, \[Dm]dálky \[Am]volají
\[Am]Vítr už příhodný \[G]vane \[Am]nám
\[Am]Tajemné příběhy \[Dm]{nás teď} \[Am]čekají
\[Am]Tvým domovem bude \[G]oce\[Am]án
\chordsoff
\cl\ifchorded\chordsoff\fi
\bverse
\chordson
\[Am]{V ráhno}\[Em]{ví plachty} \[Am]vítr \[Em]nadouvá
\[Am]Žene \[G]loď v širou \[Em]dál
\[Dm]Kolé\[F]bá boky \[Am]plachetnice
\[Dm]{Jak by si} \[F]{s ní} jenom \[E\hidx{7}]hrál
\chordsoff
\cl\ifchorded\chordsoff\fi
\cverse
Posádku ani škuner neleká
Bouře ni uragán
Přítomnost země oznámí nám
\chordson
\[Dm]Příletem \[F]kormo\[Am]rán
\chordsoff
\cl\ifchorded\chordsoff\fi
\averse
Náš ostrov vzdálený z vln se vynoří
Z příboje snů našich pustý kraj
Zátoku písčitou úsvit odhalí
Háj palem, útesy bílých skal
\cl\ifchorded\chordsoff\fi
\bverse
Příď krájí vlny i tvůj čas
Srdce tvé tluče rázně
Nástrahy moře, nebezpečí
S přáteli zvládneš vždy snáz
\cl\ifchorded\chordsoff\fi
\cverse
V přátelství najdeš pevnou hráz
Zbaví tě smutku, bázně
Zítra, až naše cesta skončí
Staneš se jedním z nás
\cl\ifchorded\chordsoff\fi
\endsong

\beginsong{Zatanči}[by={Jaromír Nohavica}]\hypertarget{song-31}{}\label{song-31}
\emptyv
\cseq{\[Em] \[G] \[D] \[Em]}\\
\cl
\chordsoff
\num
Zatanči, má milá, zatanči pro mé oči
Zatanči a vetkni nůž do mých zad
Ať tvůj šat, má milá, ať tvůj šat na zemi skončí
Ať tvůj šat, má milá, rázem je sňat
\fin
\chor
Zatanči, jako se okolo ohně tančí
Zatanči jako na vodě loď
Zatanči jako to slunce mezi pomeranči
Zatanči a pak ke mně pojď
\cl
\num
Polož dlaň, má milá, polož dlaň na má prsa
Polož dlaň nestoudně na moji hruď
Obejmi, má milá, obejmi moje bedra
Obejmi je pevně a mojí buď
\fin
\repchorus{\emptyspace}
\num
Nový den než začne, má milá, nežli začne
Nový den než začne, nasyť můj hlad
Zatanči, má milá, pro moje oči lačné
Zatanči a já budu ti hrát
\fin
\repchorus{\emptyspace}
\repchorus{\emptyspace}
\endsong


\end{songs}
\end{document}
