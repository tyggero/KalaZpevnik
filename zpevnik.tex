\documentclass[a5paper,10pt]{book}
\usepackage[chorded]{songs}
\usepackage[czech]{babel}
\usepackage[utf8]{inputenc}
\usepackage{fullpage}
\usepackage{ifthen}
\usepackage{textpos}
\usepackage[T1]{fontenc}
\usepackage{lmodern}
\usepackage[unicode=true,hidelinks]{hyperref}

\def \nempty {999}
\def \nchorus {1000}
\def \nchorusi {1001}
\def \nchorusii {1002}
\def \naverse {1101}
\def \nbverse {1102}
\def \ncverse {1103}
\def \nintro {1201}
\def \nsolo {1202}
\def \nbridge {1203}
\def \nrecite {1204}
\newcounter{oldversenum}
\newindex{titleidx}{titleidx}
\newauthorindex{authidx}{authidx}
\songcolumns{1}
\songpos{3}
\setlength{\versenumwidth}{0.75cm}
\addtolength{\hoffset}{-15pt}
\addtolength{\voffset}{-35pt}
\addtolength{\textheight}{100pt}
\addtolength{\textwidth}{35pt}
\setlength{\sbarheight}{0pt}
\setlength{\versesep}{10pt plus 8pt minus 2pt}
\pagestyle{empty}
\renewcommand{\printversenum}[1]{\ifthenelse{\equal{\theversenum}{\nempty}}{}{\small\sffamily\ifthenelse{\equal{\theversenum}{\nchorus}}{R.}{\ifthenelse{\equal{\theversenum}{\nchorusi}}{R1.}{\ifthenelse{\equal{\theversenum}{\nchorusii}}{R2.}{\ifthenelse{\equal{\theversenum}{\naverse}}{A.}{\ifthenelse{\equal{\theversenum}{\nbverse}}{B.}{\ifthenelse{\equal{\theversenum}{\ncverse}}{C.}{\ifthenelse{\equal{\theversenum}{\nintro}}{Intro }{\ifthenelse{\equal{\theversenum}{\nsolo}}{Solo }{\ifthenelse{\equal{\theversenum}{\nbridge}}{M.}{\ifthenelse{\equal{\theversenum}{\nrecite}}{Rec.}{\theversenum.}}}}}}}}}}}\ }
\renewcommand{\printchord}[1]{\bf\sffamily#1}
\renewcommand\musicnote[1]{\ifchorded\vspace{-5pt}\textnote{#1}\vspace{-5pt}\fi}

\afterpreludeskip=0pt plus 5pt
\beforepostludeskip=0pt plus 5pt
\baselineadj=2pt plus 1pt minus 2pt
\renewcommand{\clineparams}{
  \baselineskip=10pt
  \lineskiplimit=1pt
  \lineskip=1pt
}


\renewcommand{\capo}[1]{\iftranscapos\transpose{#1}\else\musicnote{\sffamily{}Capo #1\normalfont}\fi}

\newcommand{\musicblock}[1]{\ifchorded #1 \fi}
% \newcommand{\ctabblock}[1]{\begin{textblock}{10}(5,-0.5)#1\end{textblock}}

\renewcommand{\idxtitlefont}{\small\normalfont}
\renewcommand{\idxlyricfont}{\small\normalfont}
\renewcommand{\idxauthfont}{\small\normalfont}
\newcommand{\authfont}{\itshape\sffamily}
\renewcommand{\stitlefont}{\bf\Large\sffamily}
%\renewcommand{\printsongnum}[1]{\bf\LARGE\sffamily#1}
\renewcommand{\printsongnum}[1]{}
\renewcommand{\snumbgcolor}{white}
\renewcommand{\extendprelude}{\vspace{2pt}\footnotesize\showauthors\showrefs}
\newcommand{\fadeout}{\footnotesize\sffamily to fade out \normalfont\normalsize}

\makeatletter

\renewcommand\showauthors{%
  \setbox\SB@box\hbox{\authfont\sfcode`.\@m\songauthors\normalfont}%
  \ifdim\wd\SB@box>\z@\unhbox\SB@box\par\fi%
}

% \newcommand\textsubscript[1]{\@textsubscript{\selectfont#1}}
% \def\@textsubscript#1{{\m@th\ensuremath{_{\mbox{\fontsize\sf@size\z@#1}}}}}
% \newcommand\textbothscript[2]{%
  % \@textbothscript{\selectfont#1}{\selectfont#2}}
% \def\@textbothscript#1#2{%
  % {\m@th\ensuremath{%
    % ^{\mbox{\fontsize\sf@size\z@#1}}%
    % _{\mbox{\fontsize\sf@size\z@#2}}}}}
% \def\@super{^}\def\@sub{_}

% \catcode`^\active\catcode`_\active
% \def\@super@sub#1_#2{\textbothscript{#1}{#2}}
% \def\@sub@super#1^#2{\textbothscript{#2}{#1}}
% \def\@@super#1{\@ifnextchar_{\@super@sub{#1}}{\textsuperscript{#1}}}
% \def\@@sub#1{\@ifnextchar^{\@sub@super{#1}}{\textsubscript{#1}}}
% \def^{\let\@next\relax\ifmmode\@super\else\let\@next\@@super\fi\@next}
% \def_{\let\@next\relax\ifmmode\@sub\else\let\@next\@@sub\fi\@next}

\makeatother

\newcommand{\reppart}[1]{[: #1 :]}
\newcommand{\pick}[1]{\musicnote{\sffamily\MakeUppercase{#1}\normalfont}}
\newcommand{\num}{\beginverse}
\newcommand{\fin}{\endverse}
\newcommand{\start}[1]{\setcounter{oldversenum}{\value{versenum}}\setcounter{versenum}{#1}\beginverse}
\newcommand{\cl}{\endverse\setcounter{versenum}{\value{oldversenum}}}
\newcommand{\repsec}[2]{\start{#1} #2\\ \cl}
\newcommand{\emptyv}{\start{\nempty}}
%\newcommand{\emptyv}{\vspace{-\versesep}\start{\nempty}}
\newcommand{\freev}{\start{\nempty}}
\newcommand{\emptyspace}{\hspace{1pt}}
\newcommand{\chor}{\start{\nchorus}}
\newcommand{\intro}{\start{\nintro}}
\newcommand{\solo}{\start{\nsolo}}
\newcommand{\bridge}{\start{\nbridge}}
\newcommand{\chorusi}{\start{\nchorusi}}
\newcommand{\chorusii}{\start{\nchorusii}}
\newcommand{\averse}{\start{\naverse}}
\newcommand{\bverse}{\start{\nbverse}}
\newcommand{\cverse}{\start{\ncverse}}
\newcommand{\recite}{\start{\nrecite}}
\newcommand{\repchorus}[1]{\repsec{\nchorus}{#1}}
\newcommand{\repchorusi}[1]{\repsec{\nchorusi}{#1}}
\newcommand{\repchorusii}[1]{\repsec{\nchorusii}{#1}}
\newcommand{\chords}{\beginverse*}
% \newcommand{\repchorusadd}[1]{\beginchorus #1\\ \endchorus}
\newcommand{\cseq}[1]{\vspace{-\versesep}{\nolyrics #1}}
\newcommand{\ctab}[3]{\flushright\gtab{#1}{#2}{#3}}
\newcommand{\hidx}[1]{\textsuperscript{#1}}
\newcommand{\didx}[1]{\textsubscript{#1}}
\renewcommand{\rep}[1]{\sffamily\footnotesize($\times$#1)\normalsize\normalfont}
% \newcommand{\hidx}[1]{$^{\mathrm{#1}}$}
% \newcommand{\didx}[1]{$_{\mathrm{#1}}$}
% \authsepword{ a }

% \usepackage[pdftex]{hyperref}
% \hypersetup{colorlinks=false}

\begin{document}
\sffamily
~\vspace{\stretch{1}}
\begin{center}
\Huge{}\textbf{Superzpěvník}\normalsize\\[5ex]
cokoliv sebrané odkudkoliv\\[1ex]
bez garance správnosti\\[1.5ex]
\copyright~Jan Šimbera - Prófa, simbera.jan@gmail.com\\[1.5ex]
volně šiřitelné pod licencí Creative Commons BY-NC-SA
\normalfont
\end{center}
\vspace{\stretch{2}}
\rmfamily
\newpage
\showindex[2]{Písně podle názvu}{titleidx}
\newpage
\begin{songs}{}
\beginsong{Anděl}[by={Karel Kryl}]\hypertarget{song-1}{}\label{song-1}
\emptyv
\cseq{\[C] \[Am] \[C] \[G\hidx{7}]}\\
\cl
\chordsoff
\num
Z rozmlácenýho kostela v krabici s kusem mýdla
Přinesl jsem si anděla, polámali mu křídla
Díval se na mě oddaně, já měl jsem trochu trému
Tak vtiskl jsem mu do dlaně lahvičku od parfému
\fin
\chor
A proto prosím, věř mi, chtěl jsem ho žádat
Aby mi mezi dveřmi pomohl hádat
\chordson
\[C]{Co mě} čeká \[Am]   \[G\hidx{7}]{a nemi}\[C]ne
\[C]{Co mě} čeká \[Am]   \[G\hidx{7}]{a nemi}\[C]ne
\cl
\num
Pak hlídali jsme oblohu, pozorujíce ptáky
Debatujíce o Bohu a hraní na vojáky
Do tváře jsem mu neviděl, pokoušel se ji schovat
To asi ptákům záviděl, že mohou poletovat
\fin
\repchorus{\emptyspace}
\num
Když novinky mi sděloval u okna do ložnice
Já křídla jsem mu ukoval z mosazný nábojnice
A tak jsem pozbyl anděla, on oknem odletěl mi
Však přítel prý mi udělá novýho z mojí helmy
\fin
\repchorus{\emptyspace}
\endsong

\beginsong{Andělská}[by={Zuzana Navarová}]\hypertarget{song-2}{}\label{song-2}
\num
\[A]Bože, \[Hm\hidx{7}]podej mi ještě ten \[A\hidx{maj7}]kalich \[Hm\hidx{7}]
To \[A]víno, co \[Hm\hidx{7}/F\shrp{}]{s nocí se měnívá} \[C\shrp{}m\hidx{7}]{na líh} \[C\shrp{}\hidx{7}]
Snad \[F\shrp{}m]{nad ránem} na dně se \[F\shrp{}m/maj]dočtu
že \[F\shrp{}m\hidx{7}]{v kastlíku} andělskou \[D\shrp{}m\hidx{7/5b}]{počtu mám} \[D\hidx{maj7}]
\[Hm/F\shrp{}]{Tu ru tu} \[E]tu
\fin\chordsoff\num
A že hoře se skutálí do jedné slzy
Ta slza se vsákne a pak ji to mrzí
Že po prvním slunci se nadechne
a kometa zamíří nad Betlém
Tu ru tu tu~-- zatím
\fin\chorusi
\chordson
\[A]Počítám \[Hm\hidx{7}]nebe, \[C\shrp{}m\hidx{7}]světla, kam \[Hm\hidx{7}]nedostanu
\[A]Miluju \[C\shrp{}m\hidx{7}]{tebe a pak} \[Hm\hidx{7}/F\shrp{}]{usnu a} \[E]vstanu~-- a tak
\[A]Tu \[Hm\hidx{7}]{tu tu} \[C\shrp{}m\hidx{7}]tu \[Hm\hidx{7}]
\cseq{\[A] \[C\shrp{}] \[Hm] \[E]}
\chordsoff
\cl\num
Tak podej, podej mi ještě ten kalich
To víno, co s nocí se měnívá na líh
Snad na dně se perlama zaleskne
Pak vyměním veselé za teskné~-- písně
Tu ru tu tu
\fin\num
A pak hoře se skutálí do jedné básně
Ta báseň se vsákne a na zemi zasně-ží
A po prvním slunci a s první tmou
Ta kometa zamíří nad Letnou
Tu ru tu tu~-- zatím
\fin\chorusii
Počítám nebe, pusy, co nedostanu\ldots
\cl
\repsec{1}{Tak podej\ldots}
\repchorusi{Počítám nebe\ldots a tak-}
\repchorusii{Počítám nebe\ldots}
\musicnote{\fadeout}
\endsong

\beginsong{Batalion}[by={Spirituál kvintet}]\hypertarget{song-3}{}\label{song-3}
\bridge
\[Em]Víno \[G]máš a \[D]marky\[Em]tánku, \[G]dlouhá noc \[D]se \[Em]pro\[Hm]hý\[Em]ří
Víno \[G]máš a \[D]chvilku \[Em]spánku, \[G]díky, dí\[D]ky, \[Em]ver\[Hm]bí\[Em]ři
\cl
\chordsoff
\num
\chordson
\[Em]Dříve, než se rozední, kapitán \[G]{k osedlání} \[D]rozkaz \[Em]dá\[Hm]vá
\[Em]Ostruhami do slabin \[G]ko\[D]ně \[Em]po\[Hm]há\[Em]ní
Tam na straně polední čekají \[G]ženy, zlaťá\[D]{ky a} \[Em]slá\[Hm]{va}
\[Em]{Do výstřelů} karabin \[G]zvon \[D]už \[Em]vy\[Hm]zvá\[Em]{ní}
\fin
\chor
\chordson
\[Em]Víno na ku\[G]ráž a \[D]pomilovat marky\[Em]tánku
Zítra do Bur\[G]gund bata\[D]lion \[Em]za\[Hm]mí\[Em]ří
Víno na ku\[G]ráž a \[D]{k ránu} dvě hodiny \[Em]spánku
Díky, díky \[G]vám, králov\[D]ští \[Em]ver\[Hm]bí\[Em]ři
\cl
\num
Rozprášen je batalion, poslední vojáci se k zemi hroutí
Na polštáři z kopretin budou věčně spát
Neplač, sladká Marion, verbíři nové chlapce přivedou ti
Za královský hermelín padne každý rád
\fin
\repchorus{\emptyspace}
\bridge\emptyspace\\ \cl
\endsong

\beginsong{Buď můj salám}[by={Hm}]\hypertarget{song-4}{}\label{song-4}
\begin{textblock}{5}(9,-0.5) \gtab{E\hidx{sus4}}{002200:001200} \end{textblock}
\emptyv
\[D]Nemohu dnes \[E\hidx{sus4}]usnout, neboť \[Em\hidx{7}]{raší ze mě} \[A\hidx{7}]den
\[D]Zvláštní pan \[E\hidx{sus4}]sen \[Em\hidx{7}]nepřivře má \[B]víčka do dve\[A\hidx{7}]ří
Nemohu \[B]spát, tvá \[A]krása nene\[B]chá mě chladným
\[Hm]{Jen tuhle} \[G]noc a \[Hm]{pak už} budu \[A]naříkat
\[D]Já nemám \[Hm]hlad, nechci \[G]pít, nechci \[A]snídat
\[D]Já mám břicho \[Hm]plné i \[G]srdce plné, i \[B]pod kůží
mám \[F\shrp{}]strach, že něžné \[G]jídlo přemě\[F\shrp{}]{ní se v nerost} \[G]drsný
\[Em]{A že mou} \[D]lásku \[G]odnese si horník \[A]na památku
\cl\emptyv
\cseq{\[D] \[E\hidx{sus4}] \[Em\hidx{7}] \[A\hidx{7}] \[D] \[E\hidx{sus4}] \[Em\hidx{7}] \[B]}\\
\[A]Že něžné \[B]jídlo přemě\[A]{ní se v nerost} \[B]drsný
\[Hm]{A že mou} \[G]lásku \[Hm]odnese si horník \[A]na památku
\[D]Ostatní \[E\hidx{sus4}]milenci jsou \[Em\hidx{7}]{jako octa} \[A\hidx{7}]pijáci
Kdežto \[D]já mám doma \[E\hidx{sus4}]whisky
Mám \[Em]bourbon, mám \[B]curacao, mám \[A]gin,
mám \[B]cognac, mám \[A]tebe, mám \[B]salám
\[Hm]Mám tě \[G]rád~-- \[Hm]buď můj \[A]salám, lásko
\[D]Mám tě \[Hm]rád~-- \[G]buď můj \[A]salám, lásko
\[D]Mám tě \[Hm]rád~-- \[G]buď \[B]můj sa\[F\shrp{}]lá\[G]á\[F\shrp{}]á\[G]ám
\[Em]{Ať ani} \[D]koně \[G]neumírají \[A]bezdůvodně!
\fin\ifchorded\chordsoff\fi
\endsong

\beginsong{Carpe diem}[by={AG Flek}]\hypertarget{song-5}{}\label{song-5}
\begin{textblock}{5}(9,-0.5) \gtab{G/F\shrp{}}{200033:100034} \end{textblock}
\num
\chordson
\[Am]Nadešel asi \[Dm\hidx{7}]poslední den
Podí\[G]vej, celá \[G/F\shrp{}]planeta \[Em]blázní
A já \[Am]neuroním ani \[Dm\hidx{7}]slzu pro ni
Jenom \[G]zamknu dům  \[E]
A půjdu \[Am]{po kolejích} až \[Dm\hidx{7}]{na konečnou}
Hle jak \[G]mám krok \[G/F\shrp{}]vojensky \[Em]rázný
A \[Am]nezastavím ani \[Dm\hidx{7}]{na červenou}
Natruc \[G]předpisům
\chordsoff
\fin\ifchorded\chordsoff\fi
\chor
\chordson
\[Am]Tak tady mě \[G]máš
\[Am]Dnes můžeš \[G]říkat klidně co \[C]chceš
\chordsoff
Zbylo tak málo slov
\chordson
Tak \[Am]málo vět, co \[G]nelžou
\chordsoff
\cl\ifchorded\chordsoff\fi
\num
Tak už si nebudeme hrát na román
Šetři růž, nikdo nás nenatáčí
Je poslední den a zbyla nám jen
Miska cukroví
Ať všechny hospody dnes doženou plán
Ať svět z posledního pije a tančí
Já nebudu pít, nechám naplno znít
V hlavě všechno co mám
\fin
\num
Žádný slib z těch, co jsem ti dal
Nejde vyplnit a nejde vzít spátky
Tak ať točí se svět mladší o deset let
Na desce Jethro Tull
Ať platí aspoň dnes co dřív jsem jen lhal
Carpe diem, život je krátký
V tvých očích je klid
A nemám chuť snít, co by bylo dál
\fin
\endsong

\beginsong{Cestou do Jenkovic}[by={Radůza}]\hypertarget{song-6}{}\label{song-6}
\musicnote{akordy na konci sloky}
\intro
\cseq{\[Hm] \[A] \[G] \[A]}
\cl
\num
\[D]Můj děda z kola \[Hm]seskočil \[C]před prázdnou kašnou \[B]{na ná}\[C]{městí}
\[D]{Na lavičce} chleba \[Hm]posvačil, \[C]seřídil hodinky \[B]{na zá}\[C]{pěstí}
\fin
\chordsoff
\chor
\chordson
\reppart{A \[D]čápi z komína \[A]{od cihelny}
\[C]Zobákem klapou, asi jsou \[G]nesmrtelný}
\cl
\num
Tři kluci v bílejch košilích dělili se o poslední spartu
Ze zídky do záhonu skočili, přeběhli ulici a zmizeli v parku
\fin
\repchorus{\emptyspace}
\num
V oknech svítěj peřiny, na bílý kafe mlíko se vaří
Teď právě začaly prázdniny, venku je teplo a všechno se daří
\fin
\chor
\fadeout
\cl
\endsong

\beginsong{Čas}[by={Tomáš Klus}]\hypertarget{song-7}{}\label{song-7}
\emptyv
\cseq{\[Hm] \[G] \[D] \[A]}\\
\cl
\chordsoff
\num
V jednu chvíli oba zavřem oči
Asi proto, že už není víc co říct
Je to zklamání toho, kdo právě procit
Tyhle dokonale šťastný konce
Slunce zrovna přiválo jaro
Na rovných cestách křivý stíny
A cizí masky a manekýny a slova lásky
Můj sladký život mezi světy
Mý slepý rány bez odvety
A nepochopitelná touha plout
Raděj' shořet, než vyhasnout!
\fin
\num
Když má člověk svý usměvavý stavy
Celej svět se baví s ním
Zatímco chceš-li brečet, musíš, člověče, sám
Mám strach, že mezi náma
Hloubíš propast do neznáma
Že se někde ve mně snažíš nadechnout
Raděj' shořet, než vyhasnout!
\fin
\num
Za oknem do dvora kouř cigarety
Malý útěk před tím, co ještě přežívá
Stačí slovo, proč dlouhý věty, když víš
Že se Bůh stejně nedívá
A tak se kolem nás stahuje neúprosný čas
A střepy loňských let poskládám a chci zpět
Ale už nemůžu se hnout
Raděj' shořet než vyhasnout!
\fin
\num
Bojím se malých zaváhání
A všeho, co se očekává
V dlani se uložit k spánku a k ránu vstát
Schovat se všemu a všem
Jen tak probloudit se osudem~-- oslněn
Být na chvíli sám sebou
Žít sladký život mezi světy
Dávat slepý rány bez odvety
A mít nepochopitelnou touhu plout
Raděj' shořet než vyhasnout!
\fin
\endsong

\beginsong{Čert ví proč}[by={Zuzana Navarová}]\hypertarget{song-8}{}\label{song-8}
\transpose{7}
\num
Kdo snídá \[C]špek a \[F]tlustě si ho krájí\[C]
Kdo pořád slídí\[F]{ a nic} nechce nají\[C]t
Kdo usne \[G]jenom když má v dlani \[F]minci
Kdo \[B\flt]{z hlavy} má jen stojan \[F]{na čepici}
\[G]Přitom \[C]{v nebi} chtěl by housle \[F]hrát
Ten \[Am]{bude v} pekle \[G]bubno\[F]vat
\fin
\chordsoff
\num
\chordson
Kdo \[G]vždycky \[C]pro nic \[F]za nic ztropí povyk\[C]
\chordsoff
Kdo pořád mluví a nic nevysloví
Kdo za dvě zlatky vyměnil by duši
Kdo hlavu má jen na nošení uší
A přitom v nebi chtěl by housle hrát
Ten bude v pekle bubnovat
\fin
\chor
\chordson
\[D]Nekrmte čerty, \[F]{co se} krčí v křoví
\[C]Stejně vám \[Am]nikdy, \[F]nikdy neod\[G]poví
Proč \[Am]{svět je} jako kolo\[F]toč
A jenom \[G]řeknou~-- čert ví \[C]proč
\cl
\num
Kdo ze všech dveří poníženě couvá
Kdo kozí nohu pod peřinu schoval
Kdo v hlavě nemá a má pořád v gatích
Kdo se ti směje, když jsi srdce ztratil
A přitom v nebi chtěl by housle hrát
Ten bude v pekle bubnovat
\fin
\repchorus{\emptyspace}
\endsong

\beginsong{Čtyři slunce}[by={Vypsaná fixa}]\hypertarget{song-9}{}\label{song-9}
\emptyv
\cseq{\[Em] \[C] \[G] \[D]}\\
\cl
\chordsoff
\num
\chordson
\[Em]Čtyři \[C]slunce \[G]svítěj \[D]pro radost \[Em]    \[C]  \[G]  \[D]
\[Em]Vlnov\[C]{ka se} \[G]zvedne, \[D]když má dost \[Em]    \[C]  \[G]  \[D]
\fin
\chor
\chordson
\[Em]Čtyři slunce \[C]svítěj, li\[Am]{dem pod} nima \[D]není zima
Jsi \[Em]{jako malý} \[C]dítě a \[Am]můžeš začít \[D]zase znova
Já \[Em]vím, \[C]někdy to \[Am]nejde \[D]  \[D\didx{sus4}]
Já \[Em]vím, \[C]všechno to \[Am]přejde \[D]  \[D\didx{sus4}]
\cl
\num
Čtyři slunce svítěj pro radost
Přejdu řeku, vede přes ní most
\fin
\repchorus{\emptyspace}
\num
Čtyři slunce svítěj pro radost
Vlnovka se zvedne, když máš dost
\fin
\chor
\ldots{} Já vím, někdy to nejde
Já vím, všechno to přejde
\cl
\num
Čtyři slunce svítěj pro radost
Vlnovka se zvedne, když máš dost
\fin
\endsong

\beginsong{Darmoděj}[by={Jaromír Nohavica}]\hypertarget{song-10}{}\label{song-10}
\num
\[Am]{Šel včera} městem \[Em]{muž a} šel po hlavní \[Am]třídě \[Em]    \[Am]
Šel včera městem \[Em]{muž a} já ho z okna \[Am]viděl \[Em]
\[C]{Na flétnu} chorál \[G]hrál, znělo to jako \[Am]zvon
A byl v tom všechen \[Em]{žal, ten} krásný dlouhý \[F]tón
A já jsem náhle \[F\shrp{}\didx{dim}]{věděl: Ano,} to je \[E\hidx{7}]{on, to} je \[Am]{on}
\fin
\chordsoff
\num
Vyběh jsem do ulic jen v noční košili
V odpadcích z popelnic krysy se honily
A v teplých postelích lásky i nelásky
Tiše se vrtěly rodinné obrázky
A já chtěl odpověď na svoje otázky, otázky
\fin
\chor
\chordson
\[Am]{Na na} na \[Em]na\ldots{} \[C]  \[G]  \[Am]    \[F]  \[F\shrp{}\didx{dim}]      \[E\hidx{7}]
\cl
\num
Dohnal jsem toho muže a chytl za kabát
Měl kabát z hadí kůže, šel z něho divný chlad
A on se otočil a oči plné vran
A jizvy u očí, celý byl pobodán
A já jsem náhle věděl kdo je onen pán~-- onen pán
\fin
\num
Celý se strachem chvěl, když jsem tak k němu došel
A v ústech flétnu měl od Hieronyma Bosche
Stál měsíc nad domy jak čírka ve vodě
Jak moje svědomí, když zvrací v záchodě
A já jsem náhle věděl: to je Darmoděj, můj Darmoděj
\fin
\chor
\chordson
\[Am]{Můj Darmo}\[Em]{děj, vaga}\[C]bund osu\[G]{dů a} lásek
\[Am]{Jenž prochá}\[F]{zí všemi} \[F\shrp{}\didx{dim}]{sny,  ale} dnům \[E\hidx{7}]vyhýbá se
\cl
\freev
Můj Darmoděj, krásné zlo, jed má pod jazykem
Když prodává po domech jehly se slovníkem
\cl
\num
Šel včera městem muž, podomní obchodník
Šel, ale nejde už, krev skápla na chodník
Já jeho flétnu vzal a zněla jako zvon
A byl v tom všechen žal, ten krásný dlouhý tón
A já jsem náhle věděl: ano, já jsem on, já jsem-
\fin
\chor
Váš Darmoděj, vagabund osudů a lásek
Jenž prochází všemi sny, ale dnům vyhýbá se
Váš Darmoděj, krásné zlo, jed mám pod jazykem
Když prodávám po domech jehly se slovníkem
\cl
\endsong

\beginsong{Dva havrani}[by={Asonance}]\hypertarget{song-11}{}\label{song-11}
\num
\chordson
Když jsem se \[Dm]{z pole} \[C]vrace\[Dm]la
Dva havrany jsem \[C]slyše\[Dm]la
Jak jeden \[F]druhého se \[Dm]ptá\[C]á:
\reppart{\uv{\[Dm]Kdo dneska veče\[C]{ři nám} \[Dm]dá?}}
\chordsoff
\fin\ifchorded\chordsoff\fi
\num
Ten první k druhému se otočil
A černým křídlem cestu naznačil
Krhavým zrakem k lesu hleděl
\reppart{A takto jemu odpověděl:}
\fin\ifchorded\chordsoff\fi
\num
Za starým náspem v trávě schoulený
Tam leží rytíř v boji raněný
A nikdo neví, že umírá
\reppart{Jen jeho kůň a jeho milá}
\fin\ifchorded\chordsoff\fi
\num
Jeho kůň dávno po lesích běhá
A jeho milá už jiného má
Už pro nás bude dosti místa
\reppart{Hostina naše už se chystá}
\fin\ifchorded\chordsoff\fi
\num
Na jeho bílé tváře usednem
A jeho modré oči vyklovem
A až se masa nasytíme
\reppart{Z vlasů si hnízdo postavíme!}
\fin\ifchorded\chordsoff\fi
\endsong

\beginsong{I cesta může být cíl}[by={Mňága a Žďorp}]\hypertarget{song-12}{}\label{song-12}
\emptyv
\cseq{\[F] \[C] \[B]}
\cl\chordsoff\num
Zrychlený vlak náhle stojí
Asi tak v půli cesty
Zbývá vzdát čest padlému stroji
A zbytek dojít pěšky
\fin\num
A náhle už není kam spěchat
Vítací výbory nebudou čekat
Kufry a příbory tu můžeme nechat
Až déšť se vsákne, tak vystoupí řeka
\fin\chor
\reppart{I cesta může být cíl} \rep{4}
\cl\num
Jediný mrak nad hlavou stojí
Nehybně~-- tak jako vážky
Jediný klas dozrává v poli
Jediným způsobem lásky
\fin
\repsec{2}{\emptyspace}
\repchorus{\fadeout}
\endsong

\beginsong{I See Fire}[by={Ed Sheeran}]\hypertarget{song-13}{}\label{song-13}
\intro
\chordsoff
Oh, misty eye of the mountain below
Keep careful watch of my brothers' souls
And should the sky be filled with fire and smoke
\chordson
Keep watching over Durin's \[Em]sons
\cl
\chordsoff
\num
\chordson
If this is to \[Em]{end in} \[G]fire
Then we should \[D]all burn \[C]together
Watch the \[Em]flames climb \[G]high \[D]into the \[Am\hidx{7}]night
Calling \[Em]{out fa}\[G]ther, oh, \[D]stand by and \[C]{we will}
Watch the \[Am\hidx{7}]flames burn \[G/B]auburn on the \[C]mountain side
\fin
\solo\emptyspace\\ \cl
\num
\chordson
And if we should \[Em]{die to}\[G]night
Then we should \[D]all die \[C]together
Raise a \[Em]glass of \[G]wine\[D] for the \[Am\hidx{7}]{last time}
Calling \[Em]{out fa}\[G]ther, oh, \[D]prepare as \[C]{we will}
Watch the \[Am\hidx{7}]flames burn \[G/B]auburn on the \[C]mountain side
Deso\[Am\hidx{7}]lation \[G/B]comes upon the \[C]sky
\fin
\chor
\chordson
Now I see \[Em]fire\[C]{, } \[D]inside the \[Em]mountain
I see \[Em]fire\[C]{, } \[D]burning the \[Em]trees
And I see \[Em]fire\[C]{, } \[D]hollowing \[Em]souls
I see \[Em]fire\[C]{, } \[D]blood in the \[Am\hidx{7}]breeze
\cl
\emptyv
And I hope that you'll remember me
\cl
\solo\emptyspace\\ \cl
\num
\chordson
Oh, should my \[Em]people \[G]fall
Then surely \[D]I'll do the \[C]same
Confined in \[Em]mountain \[G]halls
We got too \[D]close to the \[Am\hidx{7}]flame
Calling \[Em]{out fat}\[G]her, oh, \[D]hold fast and \[C]{we will}
Watch the \[Am\hidx{7}]flames burn \[G/B]{on and} on the \[C]mountain side
Deso\[Am\hidx{7}]lation \[G/B]comes upon the \[C]sky
\fin
\chor
\chordson
Now I see \[Em]fire\[C]{, } \[D]inside the \[Em]mountain
I see \[Em]fire\[C]{, } \[D]burning the \[Em]trees
And I see \[Em]fire\[C]{, } \[D]hollowing \[Em]souls
I see \[Em]fire\[C]{, } \[D]blood in the \[Am\hidx{7}]breeze
And I hope that you'll remember \[Am\hidx{7}]{me}
\cl
\bridge
\chordson
And if the \[Am\hidx{7}]night is \[Em]burning
I will \[G]cover my \[D]eyes
For if the \[Am\hidx{7}]{dark re}\[Em]turns then
My \[G]brothers will \[D]die
And as the \[Am\hidx{7}]{sky is} falling \[Em]down
It crashed \[G]into this lonely \[D]town
And with that \[Am\hidx{7}]shadow upon the ground
I \[G/B]{hear my} \[C]people screaming \[D]out
\cl
\chor
\chordson
Now I see \[Em]fire\[C]{, } \[D]inside the \[Em]mountain
I see \[Em]fire\[C]{, } \[D]burning the \[Em]trees
And I see \[Em]fire\[C]{, } \[D]hollowing \[Em]souls
I see \[Em]fire\[C]{, } \[D]blood in the \[Em]breeze
\cl
\emptyv
\chordson
I see \[Em]fire, oh you \[C]know I saw a city \[D]burning \emph{(\[Em]fire)}
I see \[Em]fire, feel \[C]the heat upon my \[D]skin \emph{(\[Em]fire)}
And I see \[Em]fire,\[C] oooooo \[D]\emph{(fi\[Em]re)}
\cl
\freev
\chordson
And I see \[Em]fire burn \[C]{on and} on the \[D]mountain \[Em]side
\cl
\endsong

\beginsong{Já chci taky bejt skaut}[by={Budějovická Osmnáctka}]\hypertarget{song-14}{}\label{song-14}
\num
\[E]Můj život s životem se nepotkal
\[A]{A když} přemejšlím o tom, jak rvát se s ním dál
\[D]Říkám si, co bych v něm neudělal teď pořádnej \[E]říz
\chordsoff
Já chci mít velkej stan, kam se zmáčkne i slon
Pak taky švejcarák, co klidně přeřízne strom
Co otevře i flašky s pitím, který maj nehoráznej říz
\fin
\chordsoff
\num
Já chci mít klóbrc, jak maj lidi na klondajku
Taky kompas, co mi najde tu nejbližší knajpu
A sichrhajcku, když si v nouzi chci udělat tábornickej špíz
Chci poznat všechny barvy všech těch značenejch cest
A taky houby, ze kterejch se můžu sject
A tu tam taky něco navíc když pudu hrát třeba á zet kvíz
\fin
\bridge
\chordson
Všem svým \[G]neřestem teď vyhlásím boj
Vezmu si \[A]skautskou zbroj, co se jí říká kroj
\cl
\chorusi
\chordson
\[E]{Já chci} bejt velkej skaut
Bejt \[G]tak trochu in a tak trochu out
Bejt \[A]připravenej na každej vzlet i pád
Drát se \[C]křovím namísto \[D]cestou stád
\cl
\cverse
\chordson
\[E]{Ať se} říká, že je život pes
Ale když \[G]bez sirek zapálíš hned celej les
To pak \[A]život hnedle začne bejt velkej raut
Když seš \[C]tak trochu in a \[D]tak trochu out
\reppart{\[G]Hej, \[A]hej, já chci taky bejt \[E]skaut}
\cl
\num
Vopéct si buřta spálenýho jako ďábel
Umět zkracovačku když si musíš zkrátit kábel
A místo fejsu používat výhradně jen morseovo kód
Místo národních menšin lovit bobry
Mít za bráchy a ségry zlý i ty dobrý
Spát ve spacáku všude, kde to přijde jen malinkato vhod
\fin
\bridge
Všem svým neřestem teď hlásím boj
Beru si skautskou zbroj, co se jí říká kroj
\cl
\repchorusi{\emptyspace}
\chorusii
Lepší je civět dál než se dívat zpět
Na jedný noze přeskotačit celej svět
Je mnohem lepší než oběma v místě stát
I když se ti škarohlídi budou smát
\cl
\bverse
\chordson
Když \[C]pomůžeš babče vynést plný smetí
To \[G]chlapům spadne čelist, ženský se na tě sletí
Páč \[C]tohle to je život, tam nemůžeš zmáčknout reset
To \[A]nedělám si vůbec p****! \emph{(Dej si deset.)}
\cl
\repchorusi{\emptyspace}
\chorusii
\ldots{} i když se ti půlka světa bude smát \ldots{}
\cl
\cverse
Ať se říká, že je život pes...
\cl
\endsong

\beginsong{Jana smutná}[by={Jananas}]\hypertarget{song-15}{}\label{song-15}
\capo{2}
\musicnote{starší verze (z živé nahrávky)}
\musicnote{mezihra má akordy jako refrén}
\num
Je \[Em]smutné, když štěňátko na nádor umírá
\chordsoff
Když prášek na spaní až ráno zabírá
\chordson
\[D]Když raněná srna \[C]padá do mlází
Když \[Em]starou paní záchranka před \[D]domem pora\[C]zí
\fin
\chordsoff
\num
Je tuze smutné, když se bílý kůň stane lidskou potravou
Když manželství zabíjí duši toulavou
Když waka-waka-é-é je textem písňovým
Když hygienik otráví se sýrem plísňovým
\fin
\chor
\chordson
\[Em]{Já vím,} \[D]každý \[C]trápení své \[Am]má
Ale \[Em]nejsmutnější \[D]{na světě} jsem \[Em]{já}
\cl
\bridge\emptyspace\\ \cl
\num
Je smutné, když ti kamarád načůrá do piva
Když Jirka Babica svou pánev zahřívá
Když půlku českých obrazů odváží si Švéd
A když tě tvůj kluk z Plzně už vůbec nemá réd
\fin
\num
Je smutné, když Petr Kotvald naživo zazpívá
A ještě smutnější, když to někdo poslouchá
Když slabý křehký racek zmírá pod naftou
Když na Slovensku přestane se blýskat nad Tatrou
\fin
\chor
\rep{2}
\cl
\bridge\emptyspace\\ \cl
\num
Je smutné, když fajn kamaráda holka ovládá
A už s ním není žádná sranda jako za mlada
Když zatoulaný pejsek do švejžužu jde
A smutnější je, když se tenhle verš do sloky nevejde
\fin
\num
Když fotbalista modelce dá náhle kopačky
A modelka pak před přehlídkou snídá kapačky
A s Karlem Gottem naopak když nikdo nesnídá
Tak to se smutné zdá, ale nejvíc trpím já
\fin
\chor
\ldots{}
Ať se nikdo nezlobí, smůla má se násobí
Já mám prostě těžší období
Ať se nikdo nediví, svět je nespravedlivý
Já jsem ten, kdo tohle nejlíp ví
Já vím, každý trápení své má
Ale nejsmutnější na světě jsem já!
\cl
\endsong

\beginsong{Jelen}[by={Jelen}]\hypertarget{song-16}{}\label{song-16}
\num
\[Dm]{Na jaře} se vrací \[C]{od podzima} li\[Dm]stí
Mraky místo ptáků \[C]krouží nad Závi\[Dm]stí
Kdyby jsi se někdy \[C]{ke mně} chtěla vrá\[Dm]tit
Nesměla bys, lásko, \[C]moje srdce ztra\[G]tit
\fin
\chordsoff
\chor
\chordson
\[Dm]Zabil jsem v lese \[C]jele\[F]na
Bez nenávisti, \[C]bez jmé\[Dm]na
Když přišel dolů k \[C]řece \[F]pít
Krev teče do vody, \[C]{v srdci} \[Dm]klid
\cl
\emptyv
\chordson
\cseq{\[Dm] \[C] \[F] \[C]} \rep{2}
\cl
\num
Voda teče k moři, po kamenech skáče
Jednou hráze boří, jindy tiše pláče
Někdy mám ten pocit, i když roky plynou
Že vidím tvůj odraz dole pod hladinou
\fin
\repchorus{\emptyspace}
\num
Na jaře se vrací listí od podzima
Čas se někam ztrácí, brzo bude zima
Svět přikryje ticho, tečka za příběhem
Kdo pozná, čí kosti zapadaly sněhem
\fin
\repchorus{\emptyspace}
\emptyv
Hej!
\cl
\chor
\emph{(o půltón výše)}
\cl
\endsong

\beginsong{Když mě brali za vojáka}[by={Jaromír Nohavica}]\hypertarget{song-17}{}\label{song-17}
\num
\[Am]{Když mě} brali za vo\[C]jáka, \[G]stříhali mě doho\[C]la
\[Dm]Vypadal jsem jako \[Am]blbec \[E]jak ti všichni doko\[F]la
\[G]La, \[C]la, \[G]la, \[Am]{jak ti} všichni \[E]doko\[Am]la
\fin
\chordsoff
\num
Zavřeli mě do kasáren, začali mě učiti
jak mám správný voják býti a svou zemi chrániti
Ti, ti, ti, a svou zemi chrániti
\fin
\num
Na pokoji po večerce ke zdi jsem se přitulil
Vzpomněl jsem si na svou milou, krásně jsem si zabulil
Lil, lil, lil, krásně jsem si zabulil
\fin
\num
Když přijela po půl roce, měl jsem zrovna zápal plic
Po chodbě furt někdo chodil, tak nebylo z toho nic
Nic, nic, nic, tak nebylo z toho nic
\fin
\num
Neplačte, vy oči moje, ona za to nemohla
Protože mladá holka lásku potřebuje, a tak si k lásce pomohla
La, la, la, a tak si k lásce pomohla
\fin
\num
Major nosí velkou hvězdu, před branou ho potkala
Řek jí, že má zrovna volný kvartýr, tak se sbalit nechala
La, la, la, tak se sbalit nechala
\fin
\num
Co je komu do vojáka, když ho holka zradila
Na shledanou, pane Fráňo Šrámku, písnička už skončila
La, la, la, jakpak se Vám líbila
La, la, la, no nic moc extra nebyla
\fin
\endsong

\beginsong{Lokomotiva}[by={Poletíme}]\hypertarget{song-18}{}\label{song-18}
\emptyv
\cseq{\[G] \[D] \[Em] \[C]}\\
\cl
\chordsoff
\num
Pokaždé když tě vidím, vím, že by to šlo
A když jsem přemejšlel, co cítím, tak mě napadlo
Jestli nechceš svýho osla vedle mýho osla hnát
Jestli nechceš se mnou tahat ze země rezavej drát
\fin
\chor
\chordson
\[G]Jsi loko\[D]motiva, kte\[Em]{rá se} řítí \[C]tmou
\[G]Jsi indi\[D]áni, kteří \[Em]prérií je\[C]dou
\[G]Jsi kulka \[D]vystřelená \[Em]{do mojí} hla\[C]vy
\[G]Jsi prezi\[D]dent a já tvé \[Em]Spojené stá\[C]ty
\cl
\num
Přines jsem ti kytku, no co koukáš, to se má
Je to koruna žvejkačkou ke špejli přilepená
A dva kelímky od jogurtu, co je mezi nima niť
Můžeme si takhle vždycky volat, když budeme chtít
\fin
\repchorus{\emptyspace}
\num
Každej příběh má svůj konec, ale né ten náš
Nám to bude navždy dojit, všude kam se podíváš
Naše kachny budou zlato nosit a krmit se popcornem
Já to každej večer spláchnu půlnočním expresem
\fin
\repchorus{\emptyspace}
\num
Dětem dáme jména Jessie, Jeddej, Jad a John,
Ve stopadesáti letech ho budu mít stále jako slon
A ty neztratíš svoji krásu, stále štíhlá kolem pasu
Stále dokážeš mě chytit lasem a přitáhnout na terasu
\fin
\chor
\rep{2}
\cl
\endsong

\beginsong{Marie}[by={Tomáš Klus}]\hypertarget{song-19}{}\label{song-19}
\emptyv
\cseq{\[C] \[E] \[F] \[G]}
\cl\chordsoff\num
Je den~-- tak pojď, Marie, ven
Budeme žít~-- a házet šutry do oken
Je dva necháme doma trucovat
Když nechtějí, nemusí~-- nebudem se vnucovat
Jémine~-- všechno zlý jednou pomine
Tak Marie~-- co ti je?
\fin\num
Všemocné jsou loutkařovy prsty
Ať jsou tenký nebo tlustý, občas přetrhají nit
A to pak jít~-- a nemít nad sebou svý jistý
Pořád s tváří optimisty listy v žití obracet
Je to jed~-- mazat si kolem huby med
A neslyšet, jak se ti bortí svět
Marie~-- kdo přežívá, nežije, tak ádijé
\fin\num
Marie~-- už zase máš k tulení sklony
Jako loni slyším kostelní zvony znít
A to mě zabije, a to mě zabije
A to mě zabije~-- jistojistě!
\fin\chor
Já mám, Marie, rád, když má moje bytí spád
Býti věčně na cestách
A k ránu spícím plícím život vdechovat
Nechtěj mě milovat
Nechtěj mě milovat
Nechtěj mě milovat!
\cl\num
Copak nemůže být mezi ženou a mužem
Přátelství, kde není nikdo nic dlužen
Prostě jen prosté spříznění duší
Aniž by kdokoli cokoli tušil
Na na na\ldots
\fin
\repchorus{\emptyspace}
\endsong

\beginsong{Mezi horami}[by={Čechomor}]\hypertarget{song-20}{}\label{song-20}
\num
\reppart{\[Am]Mezi \[G]hora\[Am]mi \[C]lipka \[G]zele\[C]ná}
\reppart{\[C]Zabili Janka, \[G]Janíčka, \[Am]Janka, \[Am]miesto \[Em]jele\[Am]ňa}
\fin\chordsoff\num
\reppart{Keď ho zabili, zamordovali}
\reppart{Na jeho hrobě, na jeho hrobě kříž postavili}
\fin\num
\reppart{Ej, křížu, křížu, ukřižovaný}
\reppart{Zde leží Janík, Janíček, Janík, zamordovaný}
\fin\ifchorded\chordsoff\fi
\num
\reppart{Tu šla Anička plakat Janíčka}
\reppart{Aj, na hrob padla a viac nevstala~-- dobrá Anička}
\fin\ifchorded\chordsoff\fi
\endsong

\beginsong{Nagasaki Hirošima}[by={Mňága a Žďorp}]\hypertarget{song-21}{}\label{song-21}
\num
\[G]Tramvají \[D]dvojkou \[C]jezdíval \[D]jsem do Žide\[G]nic \[D] \[C] \[D]
\[G]Z takový \[D]lásky \[C]většinou \[D]nezbyde \[Em]nic
\[C]Z takový \[G]lásky \[C]jsou kruhy \[G]pod oči\[D]ma
A \[G]dvě spálený \[D]srdce~-- \[C]Nagasaki \[D]Hiroši\[G]ma \[D] \[C] \[D]
\fin\chordsoff\num
Jsou jistý věci, co bych tesal do kamene
Tam, kde je láska, tam je všechno dovolené
A tam kde není, tam mě to nezajímá
Jó, dvě spálený srdce~-- Nagasaki Hirošima
\fin\num
Já nejsem svatej~-- ani ty nejsi svatá
Jablka z ráje bejvala jedovatá
Jenže hezky jsi hřála, když mi někdy byla zima
Jó, dvě spálený srdce~-- Nagasaki Hirošima
\fin
\repsec{1}{\ldots\\
\reppart{A dvě spálený srdce~-- Nagasaki Hirošima} \rep{3}}
\endsong

\beginsong{Nebezpečný síly}[by={Neřež}]\hypertarget{song-22}{}\label{song-22}
\intro
\cseq{\reppart{\[Gm] \[A] \[Dm] \[G] \[G\shrp{}\didx{dim}] \[A] \[Dm]}}
\cl\num
Co vy \[Gm]{o nás }\[A]vůbec \[Dm]víte, že prej \[Gm]každej \[A]{z nás} je \[Dm]vrah
V parku \[Gm]utrh \[A]jsem si \[Dm]kvítek \[G]bílý~-- a \[G\shrp{}\didx{dim}]{teď jsem} \[A]{v želíz}\[Dm]kách
\fin\chordsoff\num
Taky moji slabou mámu policajt mi včera vzal
\chordson
Půjči\[Gm]{la si} \[A]flašku \[Dm]{z krámu}, \[G]žíly \[G\shrp{}\didx{dim}]{bych mu} \[A]podře\[Gm]zal
\cseq{\[Dm]\[Gm]\[Dm]\[Gm]\[A]\[Dm]}
\fin\num
Jeden po kapsách má díry, druhej zlato, třetí smrt
Moje ségra balí inženýry, ale teď jí úsměv ztvrd
\fin\num
Zase udělali šťáru a ta holka měla pech
Chtěla jenom blbejch dvacet márů, za to leží na zádech
\fin\chor\chordson
\reppart{My jsme \[Gm]nebezpečný, \[G\shrp{}\didx{dim}]nebezpečný, \[B]nebez\[E7]pečný \[G\shrp{}\didx{dim}]sí\[A]ly}\\
\cseq{\reppart{\[Gm]\[A]\[Dm]\[G]\[G\shrp{}\didx{dim}]\[A]\[Dm]} \[Gm]\[Dm]\[Gm]\[Dm]\[Gm]\[A]\[Dm]}
\chordsoff\cl\num
Chlupatej si mýho brášku, co má zrovna dvanáct let
Podal kvůli barvě prášku bílý - no tak to sis poštu splet
\fin
\repsec{1}{Co vy o nás vůbec víte\ldots}
\repchorus{\emptyspace}
\num
Já jsem hajzl v líný kůži, co se nemá, musím brát
Za to, že jsem utrh bílou růži, budu nadosmrti smrad
\fin\cverse
Jako vždycky - malej smrad, línej hajzl - malej smrad
\cl
\endsong

\beginsong{Neklid}[by={Tomáš Klus}]\hypertarget{song-23}{}\label{song-23}
\emptyv
\cseq{\[C] \[E] \[Am] \[F]}
\cl\chordsoff\num
Déšť smývá barvu z plakátu na staré zdi
Co vede od tvého domu až do Klimentský
Na rohu ulice postarší prodejce knih
Už dlouho nepřišel jediný zákazník
Kouří~-- přes zákaz na dveřích
Pošťák je floutek a čte vzkazy na pohlednicích
V divadle loutek dnes hrají ochotníci
Jediná láska jejich života~-- visí
\fin\chor
Zatímco:
Ty ležíš ve vaně a snažíš se zastavit
Krev~-- teče ti po rukou a zasychá na šatech
Svázaném zápěstí páskou na koberce
A malí kluci kreslí na sklo~-- kosočtverce
\cl\bridge
Když z těla vyprchá život, stydne
Když z těla vyprchá život, je klidné
\cl\num
Národní třídu zas ovlád' slepý harmonikář
A jeho věčně spící pes a starý taxikář
Kterému vadí~-- mládí
Přes všechny prosby déšť doposud neustal
V průchodu políbil Jan ženu na ústa
Bylo to poprvé, co cítil lásku
I ona následné dotyky zpoplatní
Obnos ať dorazí na účet do dvou dní
Pak oba splynem~-- s plynem
\fin
\repchorus{\emptyspace}
\repsec{\nbridge}{\emptyspace}
\cverse
Klidné, klidné, klidné
Klidné, klidné, klidné
Klidné, klidné, klidné
Klidné, klidné, klidné
\cl
\endsong

\beginsong{Nikdy nic nebylo}[by={Sto zvířat}]\hypertarget{song-24}{}\label{song-24}
\emptyv
\cseq{\[Am] \[F] \[D] \[G]}
\cl\chordsoff\num
Nikdy jsi nebyla a naše seznámení
Proběhlo má milá před kinem, který není
Nepil jsem Tequilu~-- a ne že bych se šklebil
Netlačil na pilu v tom baru, kterej nebyl
\fin\num
Nikdy jsem neříkal, kolik mě čeká slávy
Nehrála muzika, ze který jsem se dávil
Nechtěl jsem na závěr Ti vyblejt celej život
A ztratit charakter a všechno oběživo
\fin\chordson\emptyv
\cseq{\[Am] \[C] \[D] \[Am]}
\cl\chordsoff\chor
Nebylo nic~-- já jen, kdybys měla chvíli
můžem si někam sednout
Nebylo nic~-- pár let jsme spolu nemluvili
a předtím taky ani jednou
\cl\num
Nikdy nic nebylo~-- noc jako horská dráha
Vsadím se o kilo, že prostě nejsi drahá
Je to jen chiméra~-- a žádný že jsem brečel
a měl jsem hysterák, ty už mi neutečeš
\fin\num
Scénář se nekoná~-- my dva ho nenapsali
a někdo místo nás prázdnej papír spálil
Nic není ani já, ani tvý zlatý oči
Jen jsme šli na biják, co nikdo nenatočil
\fin
\solo
\emptyspace
\cl
\repchorus{\emptyspace}
\num
Nikdy jsi nebyla a naše seznámení
Proběhlo má milá před kinem, který není
Nic není ani já, ani tvý zlatý oči
Jen jsme šli na biják, co nikdo nenatočil
\fin
\repchorus{\fadeout}
\endsong

\beginsong{Nina}[by={Tomáš Klus}]\hypertarget{song-25}{}\label{song-25}
\emptyv
\cseq{\[G] \[Em] \[C] \[D]}\\
\cl
\chordsoff
\num
Dnes v noci jsem ze spaní křičela tvoje jméno
Já vím, že nejsi rád, ale nejde zapomenout
Jak při každém slově přivíráš víčka
Prosím, vyslyš moji zpověď, už jsme starý na psaníčka
\fin
\num
Jsem bytost z vodních par, živa jenom z tvého dechu
Já vím, že nejsi rád a že je ti to k vzteku
Chci ti všechno říct a pak se někam schovat
Třeba pochopíš, jak je těžké nemilovat
\fin
\bridge\emptyspace\\ \cl
\num
Zase zrychlil se mi dech, jak maratónským běžcům
Co je to za příběh bez lásky, bez milenců
Nemáš slov, patrně všechna patří jiným
Prosím, proměň mě s nimi ve sny, v gesta i činy
\fin
\num
A já tu zůstanu, ztracené malířské plátno
Třeba se vrátíš a já zas nechám se napnout
Prosím maluj mě, tvoř k obrazu svému
Nech mě shořet, už nikdy o nás nemluv
\fin
\chor
Jsi mé úzko, jsi krev z řezných ran
Jsi ten, kdo vchází nepozván
Jsi zvuk, když padnou mi na rety slzy múz
Jsi mé úzko, jsi krev z řezných ran
Ačkoli nechci, jsi ve mě uschován
Jsi zvuk, když padnou mi na rety slzy múz
\cl
\num
Až splynu se vzduchem, nechám rozplakat nebe
Budu vším tím, co lidi k propasti svede
Budu Krysařovou flétnou a ozvěna v tvé duši
Pak ptáci tiše vzlétnou, by nedali tušit
\fin
\num
Že se nebe nakloní a zatřese světem
Tvé černé svědomí poprvé promluví k obětem
Nerovných bojů tvé sebestředné války
Srdečních nepokojů, cos pozoroval z dálky
\fin
\chor
\rep{2}
\cl
\endsong

\beginsong{Petěrburg}[by={Jaromír Nohavica}]\hypertarget{song-26}{}\label{song-26}
\num
\[Am]Když se snáší noc na střechy Petěrburgu, \[F]padá \[E]na mě \[Am]žal
Zatoulaný pes nevzal si ani kůrku \[F]chleba, kterou \[E]jsem mu \[Am]dal
\fin\chor
\reppart{\[C]Lásku moji \[Dm]kníže I\[E]gor si bere
\[F]Nad sklenkou \[D\didx{dim}]vodky \[H\hidx{7}]hraju si \[E]{s revolverem}
\[Am]Havran usedá na střechy Petěrburgu
\[F]Čert a\[E]by to \[Am]spral}
\cl\chordsoff\num
Nad obzorem letí ptáci slepí v záři červánků
Moje duše, široširá stepi, máš na kahánku
\fin\chor
\reppart{Mému žalu na světě není rovno
Vy jste tím vinna, Naděždo Ivanovno
Vy jste tím vinna
Až mě zítra najdou s dírou ve spánku}
\cl
\endsong

\beginsong{Pocity}[by={Tomáš Klus}]\hypertarget{song-27}{}\label{song-27}
\emptyv
\cseq{\[G] \[D] \[Em] \[C]}\\
\cl
\chordsoff
\freev
Z posledních pocitů
Poskládám ještě jednou úžasnou chvíli
Je to tím, že jsi tu
A možná tím, že kdysi jsme byli
Ty a já, my dva, dvě nahý těla
Tak neříkej, že jinak jsi to chtěla
Tak neříkej
Neříkej
Už neříkej mi nic
\cl
\freev
Stala ses do noci
Zničehonic moje platonická láska
Unaven, bezmocný
Usínám vedle Tebe, něco ve mně praská
\cl
\emptyv
A ranní probuzení
A slova o štěstí, neboj se, to nic není
Pohled na okouzlení
A prázdný náměstí na znamení
\cl
\chor
Jenže ty neslyšíš
Jenže ty neposloucháš
Snad ani nevidíš
Nebo spíš nechceš vidět
A druhejm závidíš
A v očích kapky slaný vody
Zkus změnu, uvidíš
A vítej do svobody
\cl
\freev
Jsi anděl, netušíš
Anděl, co ze strachu mu utrhli křídla
A až to ucítíš, zkus kašlat na pravidla
\cl
\freev
Říkej si o mě co chceš
Já jsem byl odjakživa blázen
Nevím, co nechápeš
Ale vrať se na zem
\cl
\chor
\fadeout
\cl
\endsong

\beginsong{Proměny}[by={Čechomor}]\hypertarget{song-28}{}\label{song-28}
\num
\[Am]Darmo sa ty trápíš \[G]můj milý sy\[C]nečku
Nenosím já tebe, \[E]nenosím v sr\[Am]déčku
A já tvo\[G]ja \[C]ne\[G]bu\[C]du \[Dm]ani jednu \[E]hodi\[Am]nu
\fin\chordsoff\num
Copak sobě myslíš má milá panenko
Vždyť ty jsi to moje rozmilé srdénko
A ty musíš býti má, lebo mi tě Pán Bůh dá
\fin\num
A já sa udělám malú veveričkú
A já ti uskočím z dubu na jedličku
Přece tvoja nebudu ani jednu hodinu
\fin\num
A já chovám doma takú sekérečku
Ona mi podetne dúbek i jedličku
A ty musíš býti má lebo mi tě Pán Bůh dá
\fin\num
A já sa udělám tú malú rybičkú
A já ti uplynu preč po Dunajíčku
Přece tvoja nebudu ani jednu hodinu
\fin\num
A já chovám doma takovú udičku
Co na ni ulovím kdejakú rybičku
A ty musíš býti má lebo mi tě Pán Bůh dá
\fin\num
A já sa udělám tú velikú vranú
A já ti uletím na uherskú stranu
Přece tvoja nebudu ani jednu hodinu
\fin\num
A já chovám doma starodávnú kušu
Co ona vystřelí všeckým vranám dušu
A ty musíš býti má lebo mi tě Pán Bůh dá
\fin\num
A já sa udělám hvězdičkú na nebi
A já budu lidem svítiti na nebi
Přece tvoja nebudu ani jednu hodinu
\fin\num
A sú u nás doma takoví hvězdáři
Co vypočítajú hvězdičky na nebi
A ty musíš býti má lebo mi tě Pán Bůh dá
\fin
\endsong

\beginsong{Rád chodím na poštu}[by={Pokáč}]\hypertarget{song-29}{}\label{song-29}
\chor
\[F]Rád chodím \[C]{na poštu,} \[B]nejsem tam jen \[F]{do počtu}
\[B]Jsem tam důle\[F]žitý člen \[Gm]fronty, která \[C]trčí ven
\[F]Vždy, když se \[C]cítím sám, \[B]rád na poštu \[F]zavítám
\[B]Tam je tolik \[F]lidí, že \[Gm]pohnout se \[C]mám potí\[F]že
\cl
\chordsoff
\num
\chordson
\[B]{Za přepážkou} \[F]paní v letech \[B]{má razítko} \[F]gumové
a \[B]občas tiskne \[Dm]tiskárnou z dob \[Gm]první války \[C]světové
\chordsoff
Každému, kdo spílá jí, že zas na poště prosral den
nabídne los stírací, ať uklidní se hazardem
\fin
\repchorus{\emptyspace}
\num
Proč je vždy jen jedna z pěti přepážek otevřená
Je záhada, co navždy zůstat má tajemstvím zastřená
Má nejoblíbenější služba je balíček do ruky
Znamená to totiž, že zas budu moct jít na poštu
\chordson
\[C]Jestli ho vůbec doručej, to už je bez záru\[F]ky
\fin
\chorusii
Rád chodím na poštu, nejsem tam jen do počtu
Jsem tam důležitý člen fronty, která trčí ven
Vždy když tam stepuju, tenhle song si notuju
Mám pak hnedka trochu míň chuť do mozku si vrazit klín
\cl
\endsong

\beginsong{Ráda se miluje}[by={Karel Plíhal}]\hypertarget{song-30}{}\label{song-30}
\chor
\[Am]Ráda se miluje, \[G]ráda \[C]jí, \[F]ráda si \[Em]jenom tak \[Am]zpívá
Vrabci se na plotě \[G]háda\[C]jí, \[F]kolikže \[Em]času jí \[Am]zbývá
\cl\num
\[F]Než vítr dostrká \[C]k útesu \[F]tu její legrační \[C]bár\[E]ku
\[Am]Pámbu si ve svým \[G]note\[C]su \[F]udělá \[Em]jen další \[Am]čárku
\fin\chordsoff
\repchorus{\emptyspace}
\num
Psáno je v nebeské režii, a to hned na první stránce
že naše duše nás přežijí v jinačí tělesný schránce
\fin
\repchorus{\emptyspace}
\num
Úplně na konci paseky, tam, kde se ozvěna tříští
sedí šnek ve šneku pro šneky~-- snad její podoba příští
\fin
\repchorus{\emptyspace}
\endsong

\beginsong{Řiditel autobusu}[by={The Tap Tap}]\hypertarget{song-31}{}\label{song-31}
\num
Můj novej \[C\shrp{}m]vozejk je rychlej jako vítr a silnej jako \[A]bejk
\[H]Svět s ním chutná zas jak \[G\shrp{}]propečenej stejk
\[C\shrp{}m]Život má spád, připadám si, jako bych měl aspoň metr šede\[A]sát
\[H]{A všechno} je tak jak má \[C\shrp{}m]bejt
\fin
\chordsoff
\num
To neni vozejk, mladej, tohleto je ňákej fejk
To je kolo a s tím nesmíš do autobusu
Jó, ten, kdo má kolo, tak musí jít z kola ven
Tak spánembohem, pac a pusu
\fin
\bridge
\chordson
\[G\shrp{}]Jsem řiditel tohohle autobusu, \[A]mám svý pravidla a svý \[G\shrp{}]know-how
\[G\shrp{}]{Jsi řiditel} tohohle autobusu \[A]{a je} to autobus \[G\shrp{}]{do stanice} Ouha, \[A]ou\[G\shrp{}]ha
\cl
\num
S kolem svým sem nesmíš, máš to tady černý na bílým
A nechtěj abych tu řval na celý kolo
Prostě si piš, že když s tím kolem ihned nezmizíš, tak něco uvidíš
Nestrpím tady žádnou svoloč
\fin
\num
Mám malý boty, ale práva mám úplně stejně velký jako ty
Tak na to taky myslet začni
Co máš černý na bílým, není až tak černobílý
Pro tebe je to kolo, pro mě pomůcka kompenzační
\fin
\bridge
Jsi řiditel tohohle autobusu, svět ale nejsou jízdní pruhy
Jsem řiditel tohohle autobusu, a ty mi se svym kolem neruš moje kruhy
\cl
\chor
\chordson
\[A]Kolo je \[H]kolo a \[E]kolem zůstane
Tak \[A]{to bylo} a \[H]{je a} nic se \[G\shrp{}]{na tom} nezmění
\[A]Blbec je \[H]blbec a \[E]blbcem zůstane
\[A]Někdo má krátký nohy, někdo \[H]dlouhý vedení
\chordsoff
Kolo je kolo a kolem zůstane
Z těch tvejch argumentů zůstává rozum stát
Blbec je blbec a blbcem zůstane
Když máváš předpisy, tak měl by jsi je znát
\cl
\recite
Smluvní přepravní podmínky, článek 6, odstavec 15:
Jiné pojízdné kompenzační pomůcky se považují za vozíky pro invalidy, pokud jsou svými rozměry a vahou s nimi srovnatelné!
\cl
\num
Nechci jednat kách a nijak zvlášť netoužím po hádkách
Chci jen to, na co mám nárok, na to vem jed
Lidi jako já nežijou jenom v pohádkách
Jedem v tom všichni spolu, tak nás nechte jet
\fin
\num
Pustím tě sem s kolem a má autorita bude rázem skolena
Jak to uděláš jednou, každej to pak chce
Pravidla jsou pravidla a to uzná i každej vidlák
Jestli chceš výjimku, vrať se ke Sněhurce
\fin
\bridge
Jsem řiditel tohohle autobusu, odvolávej se třeba k Bohu
Jsi řiditel tohohle autobusu, příště mi vezmeš bílou hůl nebo dřevěnou nohu
\cl
\repchorus{\emptyspace}
\endsong

\beginsong{Sáro!}[by={Traband}]\hypertarget{song-32}{}\label{song-32}
\chorusi
\[Am]Sáro, \[Em]Sáro, \[F]{v noci} se mi \[C]zdálo
Že \[F]tři andělé \[C]Boží k nám \[F]přišli na o\[G]běd
\[Am]Sáro, \[Em]Sáro, jak \[F]moc a nebo \[C]málo
Mi \[F]chybí, abych \[C]tvojí duši \[F]mohl rozu\[G]mět?
\cl\chordsoff\num
Sbor kajícných mnichů jde krajinou v tichu
A pro všechnu lidskou pýchu má jen přezíravý smích
A z prohraných válek se vojska domů vrací
Však zbraně stále burácí a bitva zuří v nich
\fin\num
Vévoda v zámku čeká na balkóně
Až přivedou mu koně, pak mává na pozdrav
A srdcová dáma má v každé ruce růže
Tak snadno pohřbít může sto urozených hlav
\fin\num
Královnin šašek s pusou od povidel
Sbírá zbytky jídel a myslí na útěk
A v podzemí skrytí slepí alchymisté
Už objevili jistě proti povinnosti lék
\fin\chorusii
Sáro, Sáro, v noci se mi zdálo
Že tři andělé k nám přišli na oběd
Sáro, Sáro, jak moc a nebo málo
Ti chybí, abys mojí duši mohla rozumět?
\cl\num
Páv pod tvým oknem zpívá sotva procit
O tajemstvích noci ve tvých zahradách
A já~-- potulný kejklíř, co svázali mu ruce
Teď hraju o tvé srdce a chci mít tě na dosah
\fin\cverse
Sáro, Sáro, pomalu a líně
S hlavou na tvém klíně chci se probouzet
Sáro, Sáro, Sáro, Sáro, rosa padá ráno
A v poledne už možná bude jiný svět!
\chordson
\[F]Sáro, \[C]Sáro, \[F]vstávej, milá \[C]Sáro
\[F]Andělé k nám \[Dm]přišli na o\[C]běd
\chordsoff
\fin
\endsong

\beginsong{Slavíci z Madridu}[by={Waldemar Matuška}]\hypertarget{song-33}{}\label{song-33}
\bridge
\cseq{\[Am] \[Em] \[H\hidx{7}] \[Em] \[Am] \[Em] \[H\hidx{7}] \[Em]}\\
\cl
\chordsoff
\num
\chordson
\[Em]{Nebe je} modrý a \[H\hidx{7}]zlatý, bílá sluneční \[Em]záře
horko a sváteční \[H\hidx{7}]šaty, vřava a zpocený tv\[Em]áře
vím, co se bude \[H\hidx{7}]dít, býk už se v ohradě vzp\[Em]íná
kdo chce, ten může \[H\hidx{7}]jít, já si dám sklenici v\[Em]ína
\fin
\chor
\chordson
\[Am]Žízeň je veliká, \[Em]život mi utíká
\[H\hidx{7}]Nechte mě příjemně \[Em]snít  \[E\hidx{7}]
\[Am]{Ve stínu} pod fíky \[Em]poslouchat slavíky
\[H\hidx{7}]Zpívat si s nima a \[Em]pít \[E\hidx{7}]
\cl
\bridge\emptyspace\\ \cl
\num
Ženy jsou krásný a cudný, mnohá se ve mně zhlídla
Oči jako dvě studny, vlasy jak havraní křídla
Dobře vím, co znamená pád do nástrah dívčího klína
Někdo má pletky rád, já radši sklenici vína
\fin
\repchorus{\emptyspace}
\bridge\emptyspace\\ \cl
\num
Nebe je modrý a zlatý, ženy krásný a cudný
Mantily, sváteční šaty, oči jako dvě studny
Zmoudřel jsem stranou od lidí, jsem jako zahrada stinná
Kdo chce, ať mi závidí, já si dám sklenici vína
\fin
\repchorus{\emptyspace}
\bridge
\fadeout
\cl
\endsong

\beginsong{Sound of Silence}[by={Simon a Garfunkel}]\hypertarget{song-34}{}\label{song-34}
\num
\[Am]Hello, darkness, my old \[G]friend
I've come to talk with you a\[Am]gain
Because a vision softl\[F]{y cree}\[C]ping
Left its seeds while I w\[F]{as slee}\[C]ping
And the \[F]vision that was planted in my \[C]brain
Still re\[Am]mains
Within the \[G]sound of \[Am]silence
\fin
\chordsoff
\num
In restless dreams I walked alone
Narrow streets of cobblestone
'Neath the halo of a street lamp
I turned my collar to the cold and damp
When my eyes were stabbed by the flash of a neon light
That split the night
And touched the sound of silence
\fin
\num
And in the naked light I saw
Ten thousand people, maybe more
People talking without speaking
People hearing without listening
People writing songs that voices never share
And no one dare
Disturb the sound of silence
\fin
\num
Fools, said I, you do not know
Silence like a cancer grows
Hear my words that I might teach you
Take my arms that I might reach you
But my words like silent raindrops fell
\chordson
\[Am]And \[C]echoed in the \[G]wells of \[Am]silence
\fin
\num
And the people bowed and prayed
To the neon god they made
And the sign flashed out its warning
In the words that it was forming
\chordson
And the sign said~-- the \[F]words of the prophets
Are \[Am]written on the subway \[C]walls and tenement \[Am]halls
And whispered in the \[G]sounds of \[Am]silence
\fin
\endsong

\beginsong{Studený nohy}[by={Radůza}]\hypertarget{song-35}{}\label{song-35}
\num
\[Em]Prší, \[Hm] \[Am]choulím se \[Hm]{do svrchníku}
\[Em]{Než se} \[Hm]otočím \[C]{na pod}\[D]patku
\chordsoff
Zalesknou se světla na chodníku
Jak pětka na věčnou oplátku
\fin
\chordsoff
\num
Slyším kroky zakletejch panen
To je vínem, to je ten pozdní sběr
Každá kosa najde svůj kámen
To je vínem, ber mě, ber
\fin
\chorusi
\chordson
Studený \[C]nohy \[H]schovám doma \[Em]{pod peřinou}
A ráno \[C]kafe dám si \[H]hustý jako \[Em]tér
Přežiju \[C]tuhle nedě\[H]{li tak} jako \[Em]každou jinou
\ldots{} Na koho \[C]slovo padne, \[H]ten je soli\[A]tér
\cl
\num
Broukám si píseň o klokočí
Prší a dlažba leskne se
Je chladno a hlava, ta se točí
Jak světla na plese
\fin
\chorusi
\rep{2}
\cl
\num
Tak mám a nebo nemám kliku
Zakletá panna směje se
A moje oči, lesknou se na chodníku
Jak světla na plese
\fin
\chorusii
\chordson
\ldots{} Na koho \[C]slovo padne, \[H]ten je soli\[Em]tér
\cl
\chorusii
\rep{2}
\cl
\endsong

\beginsong{Tak dej groš zaklínači}[by={Joey Batey}]\hypertarget{song-36}{}\label{song-36}
\freev
\chordsoff
inál CAPO I.
\cl
\chordsoff
\freev
] Jak já skromný [Dmi] bard
ěn na sto [F] krát
 s Geraltem z [G] Rivie
 [E] do písně [Ami] dát
\cl
\freev
 bílý vlk se [Dmi] pral
blem stříbr [F] ným
 armádou el [G] fů
al [G] jít [E] světem
\cl
\freev
] Po mě skoči [Dmi] li
rda přešel smích [F]
skali loutnu [G]
po zubech mých [Ami]
\cl
\freev
 ďáblo [Dmi] vy
tili mi [F] hřbet
klínač za [G] řval:
 ho a hned [E]
\cl
\freev
] Tak dej [E] groš Zaklí [C] nači
Dmi] tak nebuď [Ami] skoupý
ak nebuď skoupý oouou
dej [E] groš Zaklí [C] nači
Dmi] tak nebuď [E] skoupý
\cl
\freev
] Až na kraj světa [Dmi] jde,
se s démo [F] ny,
utí tě k [G] pláči
rmouce [Ami] ní
\cl
\freev
 zahá [Dmi] ní
ky do o [F] hrad,
zdálených [G] kopců,
velí [Ami] řád
\cl
\freev
 neřá [Dmi] de
drtíš jeho [F] hruď
idstvo jen [G] chrání
hradbou mu [Ami] buď
\cl
\freev
e příběh [Dmi] můj,
ýt šampiónem [F] zván
ť přemohl [G] zlo,
naplň mu [E] džbán
\cl
\freev
] Tak dej [E] groš Zaklí [C] nači
Dmi] tak nebuď [Ami] skoupý
ak nebuď skoupý oouou
dej [E] groš Zaklí [C] nači
ť [Dmi] lidstvo jen [E] chrání
\cl
\freev
] Tak dej [E] groš Zaklí [C] nači
Dmi] tak nebuď [Ami] skoupý
ak nebuď skoupý oouou
dej [E] groš Zaklí [C] nači
ť [Dmi] lidstvo jen [E] chrání
\cl
\freev
] Tak dej [E] groš Zaklí [C] nači
Dmi] tak nebuď [Ami] skoupý
ak nebuď skoupý oouou
dej [E] groš Zaklí [C] nači
ť [Dmi] lidstvo jen [E] chrání
\cl
\endsong

\beginsong{Tisíc dnů mezi námi}[by={Nerez}]\hypertarget{song-37}{}\label{song-37}
\averse
\[Hm]Ocelově \[Em]modrou masku \[Hm\hidx{9}]máš, \[F\shrp{}m\hidx{7}]pátý \[C\shrp{}m\hidx{7}]{týden se} ti \[Hm\hidx{7}]bráním
\[Hm]Den za dnem si \[Em]{na nit} navlé\[Hm\hidx{9}]káš \[F\shrp{}m\hidx{7}]korálky s \[C\shrp{}m\hidx{7}]{vůní našich} \[Hm\hidx{7}]dlaní
\cl\bverse
Světlo a \[F\shrp{}m\hidx{7}]tma~-- tak \[C\shrp{}m\hidx{7}]to \[Cm\hidx{7}]jsem \[Hm\hidx{7}]já
Zhasni a \[F\shrp{}m\hidx{7}]{dělej, co} se dělat \[C\shrp{}m\hidx{7}]mů\[Hm\hidx{7}]že
Světlo a \[F\shrp{}m\hidx{7}]tma~-- tak \[C\shrp{}m\hidx{7}]to \[Cm\hidx{7}]jsem \[Hm\hidx{7}]já
\reppart{Jestli chceš, \[F\shrp{}m]nezůstane na mně ani \[C\shrp{}m\hidx{7}]ků\[Hm\hidx{7}]že}
Jestli chceš, \[F\shrp{}m]nezůstane na mně ani \[D\hidx{maj7}]ků\[G\hidx{maj7}]že!
\cl\chordsoff\averse
Telefonní seznam nových tváří
To je tvůj polštář pod mou hlavou
Zatržená jména v kalendáři pátý týden ve vzduchu plavou
\cl
\repsec{\nbverse}{Světlo a tma\ldots}
\emptyv
\chordson
\cseq{\[Hm] \[G] \[A] \[Hm]}
\cl
\cverse
Tisíc dnů mezi námi jako nekonečno nevypadá
Našel jsem tě kdysi ve stohu slámy
Ještě dnes mi za košili padá
\cl\cverse
Tisíckrát vlasy rozcuchaný, vlasy rozcuchaný jako křoví máš
Sladká jak hrušky planý
V hlavě mám z tebe galimatyáš
\cl\cverse
Tisíckrát přešívaný knoflíček u krku máš
A tvůj smích zvoní stejně jako hrany
V hlavě mám z tebe galimatyáš
\chordson
V hlavě mám z tebe galimaty\[G\hidx{maj7}]áš
\cl
\endsong

\beginsong{Trampská}[by={Bratři Ebenové}]\hypertarget{song-38}{}\label{song-38}
\num
\[Dm]Mlhavým ránem bosi jdou, kanady vržou na nohou
a dálka tolik vzdále\[G]{ná je} \[Dm]blízká
\fin
\chordsoff
\emptyv
Město jsi nechal za zády, zajíci dělaj rošády
\chordson
a v křoví někdo tiše \[G]Vlajku \[Dm]píská
\[G]Najednou připadáš si ňák príma, svobodnej, a tak
Tak \[Dm]{ňák, tak} \[G]ňák, ňák \[Dm]tak
\cl
\chor
\chordson
Pojď \[F]dál, s náma se nenudíš, pojď dál, ráno se probudíš
a vedle sebe máš o šišku \[Dm]víc
Pojď \[F]dál, pod sebou pevnou zem, pojď dál, a Číro s Melounem
a Meky, Miky, Vrt, a dál už \[G]nic~-- dál \[Dm]nic
\cl
\num
Mlhavý ráno za tratí, u cesty roste kapratí
sbalíš si deku, spacák, celtu, pytel
A kdyby ňákej úředník začal ti říkat, co a jak
sbalíš si deku, spacák, celtu, pytel
Důležitý je to, co jseš, odkud jsi přišel a kam jdeš
Co jseš, kam jdeš, co jseš
\fin
\repchorus{\emptyspace}
\endsong

\beginsong{Už to nenapravím}[by={Jaroslav Samson Lenk}]\hypertarget{song-39}{}\label{song-39}
\capo{1}
\chordsoff\chor
Vap tada dap\ldots
\cl\chordson\averse
V \[Hm]devět hodin dvacet pět mě \[E]opustilo štěstí
Ten \[G]vlak, co jsem jím měl jet, na koleji \[F\shrp{}]dávno \[F\shrp{}\hidx{7}]nestál
V \[Hm]devět hodin dvacet pět \[E]jako bych dostal pěstí
Já \[G]za hodinu na náměstí měl jsem \[F\shrp{}]stát
Ale v \[F\shrp{}\hidx{7}]jiným městě
\cl\bverse
Tvá \[H\hidx{7}]zpráva zněla prostě a byla tak krátká
Že \[Em]stavíš se jen na skok, že nechalas' mi vrátka
\[A]Zadní otevřená, \[F\shrp{}]zadní otevřená
Já \[H\hidx{7}]naposled tě viděl, když ti bylo dvacet
\[Em]{Ty jsi} tenkrát řekla, že se nechceš vracet
\[A]{Že jsi} unavená, \[F\shrp{}]{ze mě} unavená
\cl\chordsoff
\repchorus{Vap tada dap\ldots}
\averse
Já čekala jsem, hlavu jako střep~-- a zdálo se, že dlouho
Může za to vinný sklep, že člověk často sleví
Já čekala jsem, hlavu jako střep, s podvědomou touhou
Já čekala jsem dobu dlouhou víc než dost
Kolik přesně, nevím
\cl\bverse
Pak jedenáctá bila a už to bylo passé
Já dřív jsem měla vědět, že vidět chci tě zase
Láska nerezaví, láska nerezaví
Ten list, co jsem ti psala, byl dozajista hloupý
Byl odměřený moc, na vlídný slovo skoupý
Už to nenapravím, už to nenapravím
\cl
\repchorus{Vap tada dap\ldots}
\endsong

\beginsong{V myčce}[by={Hm}]\hypertarget{song-40}{}\label{song-40}
\intro
\reppart{\[F]Óóó\[Cm]{\ldots{}}\[D\shrp{}]{  }}
\cl
\chordsoff
\chor
\chordson
Tak \[F]hluboko jsem ve vlastních sračkách pono\[Cm]řen
že nevidím \[D\shrp{}]ven, ani když umyješ má \[F]víčka ironem
Za tlustým umaštěným \[Cm]sklem zas uvidím jen
\[D\shrp{}]{jak hluboko} jsem pono\[G\shrp{}]řen \[B]  \[C]
\cl
\num
\chordson
\[C]Včera jsem zabil svoji \[Gm]lásku \[B]
\[C]Byla mi nevěrná \[Gm]{s chlapem,} co chodí \[B]doma v trenýr\[C]kách
\chordsoff
Dala mu všechno, o čem já jsem
jenom tak snil, když jsem ji pozdě večer na kanape stáh
\fin
\num
Možná jsem prostě jenom naivní snílek
kterej si myslel, že to všechno bude zase jako dřív
Večer si zapálím a nasadím brýle
A potmě před zrcadlem tiše jako Elvis zazpívám
\fin
\chor
Tak hluboko jsem\ldots{}
\cl
\num
Už nechci nikdy vidět slzy
které jak vojáci se řadily na okraji Tvých řas
Už nechci slyšet~-- mě to mrzí
Abych pak zjistil, že hned další týden šla jsi za ním zas
\fin
\num
Blíží se večer, už se pomalu stmívá
Protější dům se v šedém moři chladných stínů potápí
Já sedím u okna a do zdi se dívám
A na tu zeď si film pro pamětníky s tebou promítám
\fin
\chor
Tak hluboko jsem\ldots{}
\cl
\cverse
\chordson
\[F]Miláčku, všechny naše ideje \[Cm]uklidíme do \[D\shrp{}]krabice z Ikeje
\chordsoff
Vždyť se vlastně vůbec nic neděje, nádobí se ušpiní a pak se umeje \emph{(v myčce)}
Miláčku vždyť se vůbec nic neděje, nádobí se ušpiní a pak se umeje \emph{(v myčce)}\ldots{}
\cl
\endsong

\beginsong{V pořádku}[by={Nerez}]\hypertarget{song-41}{}\label{song-41}
\emptyv
\cseq{\[C] \[G] \[Am] \[E] \[F] \[C] \[G] \[C]}
\cl\chordsoff\num
Za kopci, lesy a řekami, před lety své ano řekla mi
Princezna krásná jak z reklamy, teď jsem po čepcem
Po svatbě mi skončila pohádka, sklopil jsem zpětná zrcátka
Princezna povila robátka a já se stal pitomcem
\fin\num
U srdce podivný bodání ode dne, co jsme oddáni
Nemám už výhodu podání, děravý kapsy, nic víc
Prateta, tchýně a zeťáček, střepy pro štěstí a smetáček
Život je věc plná zatáček, tak zpívám z plných plic
\fin\chor
Že prý pod čepcem je zaslíbená zem
Tam podlahu mají z papíru a tou jsem propad sem
V krajích pod čepcem je zaslíbená zem
Hledáme lásku na míru a nikdy nenajdem
\cl\num
Zasel jsem ředkvičku do řádku, manželka řekla: v pořádku
Takhle tu žijeme od pátku, sedmkrát za týden
Našel jsem v lese pět masáků, manželka řekla: prasáku!
Červi nám lezou z mrazáku, letěl jsem za nima ven
\fin
\repchorus{\emptyspace}
\num
Na krk si navlíknu oprátku, manželka řekne: v pořádku
Život si nepustíš pozpátku, neřekneš ani švec
Až budu ležet v penále, manželka řekne: Zdenále
Projel jsi poslední finále, konečně spadla klec
\fin
\repchorus{\ldots \reppart{hledáme lásku na míru a nikdy nenajdem}}
\endsong

\beginsong{Vlaštovky}[by={Traband}]\hypertarget{song-42}{}\label{song-42}
\num
\[Dm]Každé jaro \[Am]{z velké} dáli \[B]vlaštovky k nám \[F]přilétaly
\[Dm]Někdy až \[C]dovnitř do stave\[Am]{ní}
\chordsoff
Pod střechou se uhnízdily a lidé, kteří uvnitř žili
Rozuměli jejich švitoření
\fin
\chordsoff
\num
O dalekých krajích, hlubokých mořích, divokých řekách
Vysokých horách, které je nutné přelétnout
O nebeských stezkách, zářících hvězdách, o cestách domů
O korunách stromů, kde je možné odpočinout
\fin
\num
Jsme z míst, která jsme zabydlili, z hnízd, která jsme opustili
Z cest, které končí na břehu
Jsme z lidí i z všech bytostí, jsme z krve, z masa, z kostí
Ze vzpomínek, snů i z příběhů
\fin
\num
Jsme jako ti ptáci, z papíru draci, létáme v mracích
A pak se vracíme zpátky tam, kde připoutaní jsme
Jsme lidské bytosti z masa a kostí, jsme jenom hosti
Na tomhle světě~-- přicházíme, odcházíme
\fin
\num
A chceme mít jisto, že někde místo, že někde je hnízdo
Odkud jsme přišli a kam zas potom půjdeme spát
Že někde je domov, že někde je hnízdo, útulno, čisto
Někde je někdo, kdo čeká na nás, na návrat
\fin
\num
Tam v dalekých krajích, v hlubokých mořích, v divokých řekách
Ve vysokých horách, které je nutné přelétnout
Tam v nebeských stezkách, v zářících hvězdách
Na cestách domů, v korunách stromů, kde je možné odpočinout
\fin
\repsec{6}{\emptyspace}
\endsong

\beginsong{Zatanči}[by={Jaromír Nohavica}]\hypertarget{song-43}{}\label{song-43}
\emptyv
\cseq{\[Em] \[G] \[D] \[Em]}\\
\cl
\chordsoff
\num
Zatanči, má milá, zatanči pro mé oči
Zatanči a vetkni nůž do mých zad
Ať tvůj šat, má milá, ať tvůj šat na zemi skončí
Ať tvůj šat, má milá, rázem je sňat
\fin
\chor
Zatanči, jako se okolo ohně tančí
Zatanči jako na vodě loď
Zatanči jako to slunce mezi pomeranči
Zatanči a pak ke mně pojď
\cl
\num
Polož dlaň, má milá, polož dlaň na má prsa
Polož dlaň nestoudně na moji hruď
Obejmi, má milá, obejmi moje bedra
Obejmi je pevně a mojí buď
\fin
\repchorus{\emptyspace}
\num
Nový den než začne, má milá, nežli začne
Nový den než začne, nasyť můj hlad
Zatanči, má milá, pro moje oči lačné
Zatanči a já budu ti hrát
\fin
\repchorus{\emptyspace}
\repchorus{\emptyspace}
\endsong

\beginsong{Zmrzlinář}[by={Marsyas}]\hypertarget{song-44}{}\label{song-44}
\begin{textblock}{5}(8,-0.5) \gtab{C\hidx{add9}/B}{020010:020010} \gtab{F\hidx{maj7}}{003210:003210} \end{textblock}
\averse
Ani \[C]nejtenčí vlásek, \[C/H]{co v kartáči} \[Am]mám \[C\hidx{add9}/B]
Se vedrem \[F]nehejbá\[C]
Je to \[Dm]tím, že jen le\[G]ží
Snad \[C]sto litrů tuše a \[C/H]sirupu z jablek
Si \[Am]dnes večer na svou \[C\hidx{add9}/B]hlavu \[F]nalejt dám \[C]
Papí\[Dm]{rem ať mě} \[G]vláčí ten \[Dm]{kůň noha}\[G]tej
\cl\bverse
\[F\hidx{maj7}]{Vláčí mě} městem \[Em]{po kostkách}, \[F\hidx{maj7}]pěnou zvědavce \[Em]odhání
\[F]Každej \[C]den \[Dm]{a celou} \[G]noc
\cl\chordsoff\averse
Teď z protější hory tu rezavou obruč
Spustím do údolí, co ta udělá
Jako koule se koulí
A to je co říct, že má sílu se koulet
Vždyť už je jak já, pět let rezavá
Jako tvá stará postel, co dávno neslouží
\cl\bverse
Zbývá už jen pár měsíců, zbývá snad ještě míň
Každej den a celou noc
\cl\cverse\chordson
\[Gm]{Na hlavní} třídě stál zmrzlinář, jen pár porcí na dně hrnce měl
\[E]Zavřel svůj krám a odtáhl po zdích, co býval chloubou bulvá\[Gm]rů
\chordsoff
Z kostky dlažební se vylíhl pták, děti jazyky teď naříká
Pole mu roste, sílí, hlavou se korun stromů dotýká
Vedle v domě bydlel starej pán, kampak zmizel, asi někam jel
\chordson
\[E]Zbyla tu po něm stará \[A]hůl a alej krásnejch jablo\[G]ní
\cl
\repsec{\naverse}{Ani nejtenčí vlásek\ldots}
\repsec{\nbverse}{Vláčí mě městem\ldots}
\endsong


\end{songs}
\end{document}
